%Version 3.1 December 2024
% See section 11 of the User Manual for version history
%
%%%%%%%%%%%%%%%%%%%%%%%%%%%%%%%%%%%%%%%%%%%%%%%%%%%%%%%%%%%%%%%%%%%%%%
%%                                                                 %%
%% Please do not use \input{...} to include other tex files.       %%
%% Submit your LaTeX manuscript as one .tex document.              %%
%%                                                                 %%
%% All additional figures and files should be attached             %%
%% separately and not embedded in the \TeX\ document itself.       %%
%%                                                                 %%
%%%%%%%%%%%%%%%%%%%%%%%%%%%%%%%%%%%%%%%%%%%%%%%%%%%%%%%%%%%%%%%%%%%%%

%%\documentclass[referee,sn-basic]{sn-jnl}% referee option is meant for double line spacing

%%=======================================================%%
%% to print line numbers in the margin use lineno option %%
%%=======================================================%%

%%\documentclass[lineno,pdflatex,sn-basic]{sn-jnl}% Basic Springer Nature Reference Style/Chemistry Reference Style

%%=========================================================================================%%
%% the documentclass is set to pdflatex as default. You can delete it if not appropriate.  %%
%%=========================================================================================%%

%%\documentclass[sn-basic]{sn-jnl}% Basic Springer Nature Reference Style/Chemistry Reference Style

%%Note: the following reference styles support Namedate and Numbered referencing. By default the style follows the most common style. To switch between the options you can add or remove �Numbered� in the optional parenthesis. 
%%The option is available for: sn-basic.bst, sn-chicago.bst%  
 
%%\documentclass[pdflatex,sn-nature]{sn-jnl}% Style for submissions to Nature Portfolio journals
%%\documentclass[pdflatex,sn-basic]{sn-jnl}% Basic Springer Nature Reference Style/Chemistry Reference Style
\documentclass[pdflatex,sn-mathphys-ay]{sn-jnl}% Math and Physical Sciences Author Year Reference Style
%%\documentclass[pdflatex,sn-aps]{sn-jnl}% American Physical Society (APS) Reference Style
%%\documentclass[pdflatex,sn-vancouver-num]{sn-jnl}% Vancouver Numbered Reference Style
%%\documentclass[pdflatex,sn-vancouver-ay]{sn-jnl}% Vancouver Author Year Reference Style
%%\documentclass[pdflatex,sn-apa]{sn-jnl}% APA Reference Style
%%\documentclass[pdflatex,sn-chicago]{sn-jnl}% Chicago-based Humanities Reference Style

%%%% Standard Packages
%%<additional latex packages if required can be included here>

\usepackage{graphicx}%
\usepackage{multirow}%
\usepackage{amsmath,amssymb,amsfonts}%
\usepackage{amsthm}%
\usepackage{mathrsfs}%
\usepackage[T1]{fontenc}%
\usepackage[utf8]{inputenc}%
\usepackage[brazilian]{babel}%
\usepackage{booktabs}%
\usepackage{url}%


\begin{document}

\title{\textbf{Aprendizado de Máquina para Auditoria de Conhecimentos e Sistemas Agrícolas Tradicionais, uma revisão de escopo sob PRISMA-ScR}}

\author[1]{\fnm{Catuxe Varjção de Santana} \sur{Oliveira}}
\author[2]{\fnm{Diego Vidal} \sur{Santos}}
\author[1]{\fnm{XXXXXXXXXXX} \sur{XXXXXXXXXXX}}

\affil*[1]{\orgname{Universidade Federal de Sergipe - UFS}, \orgaddress{\country{Brasil}}}
\affil[2]{\orgname{Universidade Estadual de Feira de Santana - UEFS}, \orgaddress{\country{Brasil}}}

\abstract{Sistemas Agrícolas Tradicionais (SAT) podem ser interpretados como sistemas socioecológicos acoplados, nos quais práticas de manejo, diversidade biocultural e condições biofísicas coevoluem em escalas espaço-temporais heterogêneas, o que torna a governança da integridade socioecológica dependente de evidências auditáveis. Este estudo analisa a maturidade técnica do estado da arte em Aprendizado de Máquina para operacionalizar um gêmeo digital inferencial voltado à auditoria de serviços ecossistêmicos e de indicadores de integridade associados a SAT. Sob diretrizes PRISMA-ScR, processou-se um corpus de 244 estudos (2010–2025) via triagem automatizada de pontuação ponderada (precisão 94,2\%) e validação manual (CCI=0,87). A análise integrou meta-análise de efeitos aleatórios, estatística multivariada e avaliação de princípios FAIR. No corpus analisado, observa-se um paradoxo de generalização, no qual classificadores alcançam alta acurácia in vitro (80–100\%), enquanto testes de robustez externa sugerem degradação de desempenho (queda de 11,8\% versus 5,6\% em modelos espacialmente independentes, d=0,95). A fragmentação metodológica (Modularidade Q=0,62, heterogeneidade I\textsuperscript{2}=58\%) e a incipiente conformidade FAIR são consistentes com a presença de assimetrias epistêmicas, o que pode limitar a tradução regulatória e a escalabilidade dos modelos para contextos territoriais diversos. Os resultados indicam que a auditabilidade computacional em SAT tende a demandar a transição de classificadores estáticos para modelos adaptativos, condicionados a benchmarks de integridade, com validação espacial e temporal orientada a degradação $\leq 8\%$, explicabilidade (XAI) para marcadores socioecológicos e conformidade FAIR robusta.}

\keywords{Sistemas Agrícolas Tradicionais, Aprendizado de Máquina, Auditoria Socioecológica, Integridade Socioecológica, Rastreabilidade, Serviços Ecossistêmicos}

\maketitle

\section[Introdução]{Introdução}

Sistemas Agrícolas Tradicionais (SAT) podem ser modelados como sistemas socioecológicos acoplados, nos quais práticas de manejo, diversidade biocultural e condições biofísicas coevoluem em horizontes temporais longos, sustentando paisagens culturais e serviços ecossistêmicos sob pressões crescentes no Antropoceno \citep{Berkes2003}. 

Em um cenário de instabilidade climática e erosão de biodiversidade, a persistência desses sistemas depende de evidências auditáveis de integridade socioecológica em escalas compatíveis com a governança territorial, articulando solo, clima, biota, conhecimento local e práticas de manejo sob uma lógica de verificabilidade e transparência \citep{Levin1998ComplexAdaptiveSystems}.

A legitimidade de estratégias de salvaguarda e valorização ligadas a SAT e patrimônio imaterial tende a emergir de arranjos institucionais e comunitários co-governados, nos quais reconhecimento patrimonial, conservação e instrumentos de rastreabilidade operam como meios de prestação de contas e não como fins em si, reduzindo o tecnicismo descolado das comunidades e a folclorização comercial \citep{lauradelvalle2026}.

Ao serem compreendidos como territórios socioecológicos intrinsecamente acoplados, nos quais solo, clima, biota, cultura e instituições interagem por dinâmicas não lineares e heterogeneidades espaço-temporais, os SAT produzem simultaneamente alimentos, serviços ecossistêmicos e valores bioculturais cuja mensuração depende de escolhas metodológicas, qualidade de dados e coerência operacional entre indicadores \citep{LeFloc2016S}. A complexidade sistêmica e o caráter difuso desses acoplamentos limitam a detecção de serviços ecossistêmicos por métricas convencionais, o que fragiliza a governança de bens comuns e amplia o espaço para \textit{greenwashing} quando a retórica de sustentabilidade não é sustentada por evidências verificáveis \citep{Levin1998ComplexAdaptiveSystems,Gale2023}.

Diante das limitações do instrumental analítico clássico para monitoramento em escala de paisagem \citep{Liao2023,Weiss2020}, torna-se pertinente um arcabouço que integre observações espaço-temporais e critérios de validação compatíveis com heterogeneidade territorial \citep{Belgiu2016}. 

Nesse contexto, o Aprendizado de Máquina (ML) emerge como ferramenta promissora para operacionalizar um gêmeo digital inferencial capaz de sustentar a auditoria socioecológica de SAT, ao lidar com a não-linearidade e multiescalaridade dos dados ambientais \citep{Mountrakis2011,Spyrou2025}.

A validação operacional desse sistema beneficia-se do atendimento a critérios de auditabilidade derivados das lacunas metodológicas identificadas na literatura, buscando assegurar robustez inferencial por validação espacialmente independente e estabilidade longitudinal frente à variabilidade climática \citep{Kuhn2013}. 

A transparência causal beneficia-se de métodos de Inteligência Artificial Explicável (XAI) que permitam discriminar marcadores físico-químicos de correlações espúrias \citep{Rudin2019}. Além disso, a soberania de dados é favorecida pela adesão aos princípios FAIR e por trilhas de auditoria imutáveis, sustentando reprodutibilidade e rastreabilidade da inferência \citep{Wilkinson2021,Kshetri2014}.

Nessa arquitetura, o Aprendizado de Máquina opera como mecanismo analítico para tratar a não-linearidade de dados multiescalares \citep{Mountrakis2011}, com potencial para apoiar a soberania epistêmica das comunidades produtoras quando acoplado a governança de dados e critérios explícitos de validação \citep{Li2022KGML_ag,Santos2007Epistemologies}. Em escalas geográficas amplas, o ML pode sustentar a auditabilidade de serviços ecossistêmicos ao estabelecer uma ligação verificável entre conformidade ambiental e prêmio de mercado, mitigando assimetrias informacionais associadas a fraudes e apropriação indevida \citep{Kshetri2014DigitalDivide}.

Nesse enquadramento, o presente estudo analisa criticamente se o aparato metodológico atual de Aprendizado de Máquina reúne condições de robustez inferencial, explicabilidade e governança de dados necessárias para sustentar um gêmeo digital inferencial aplicável a SAT como sistema de auditoria de serviços ecossistêmicos e de integridade socioecológica.

\section[Materiais e Métodos]{Materiais e Métodos}

\subsection[Fluxo de Seleção]{Fluxo de Seleção Bibliográfica (PRISMA)}

O processo de seleção seguiu as diretrizes PRISMA-ScR \citep{Tricco2018b}. A busca inicial recuperou 449 registros combinados das bases Scopus e Web of Science. Após a aplicação de critérios de pontuação automatizada e verificação manual, 244 estudos foram considerados elegíveis para a síntese qualitativa e quantitativa (Figura \ref{fig:prisma}).

\begin{figure}[h]
\centering
\includegraphics[width=0.8\textwidth]{../2-FIGURAS/2-EN/prisma_flowdiagram.png}
\caption{Fluxograma PRISMA 2020 do processo de seleção de estudos.}
\label{fig:prisma}
\end{figure}

A triagem inicial do corpus foi conduzida por meio da aplicação automatizada de uma função de pontuação ponderada sobre os registros recuperados, onde $P_{final}$ representa a pontuação final de seleção, $Q_{met}$ corresponde à qualidade metodológica normalizada (0 a 1), $Q_{tem}$ expressa a relevância temática normalizada (0 a 1) e $Q_{biblio}$ denota o impacto bibliométrico normalizado (0 a 1). Posteriormente, realizou-se verificação manual para consolidação da elegibilidade. A consistência entre avaliadores na etapa manual foi quantificada por coeficiente de correlação intraclasse (CCI=0,87), enquanto o desempenho da triagem automatizada apresentou precisão de 94,2\%.

Embora $Q_{met}$ componha um critério operacional de triagem voltado a consistência metodológica mínima e completude de reporte, esse componente não constitui uma avaliação crítica formal das fontes incluídas e não foi utilizado para ponderar estimativas, ajustar variâncias ou graduar risco de viés na síntese.

\subsection[Estratégia de busca e extração]{Estratégia de busca, elegibilidade e extração de dados}

O protocolo operacional e os artefatos de reprodutibilidade, incluindo as strings de busca, o corpus bibliográfico exportado em BibTeX, os scripts de triagem e as rotinas analíticas, foram depositados publicamente no Open Science Framework (OSF) sob DOI \url{https://doi.org/10.17605/OSF.IO/J7STC}.

As fontes de informação compreenderam as bases Scopus e Web of Science. A execução mais recente da busca e da exportação ocorreu em 24 de janeiro de 2026, conforme metadados de exportação presentes nos arquivos BibTeX. A janela de cobertura do corpus foi restringida ao período de 2010 a 2025.

Não se realizaram buscas manuais complementares em websites, repositórios institucionais ou listas de referências, e não houve contato com autores para obtenção de evidências adicionais.

Foram considerados elegíveis apenas artigos publicados em periódicos indexados nas bases consultadas, desde que recuperados pelas strings dentro do intervalo temporal e com metadados suficientes para charting, incluindo ao menos título e resumo. Não se aplicou restrição explícita de idioma na etapa de busca, uma vez que a estratégia combinou termos em inglês e português, enquanto a delimitação do universo de evidências foi dada pela indexação em Scopus e Web of Science, o que implica exclusão de literatura cinzenta e de repositórios não indexados, opção adotada para maximizar rastreabilidade, padronização de metadados e reprodutibilidade do pipeline analítico.

A estratégia eletrônica foi construída para recuperar evidências em que a caracterização explícita de Sistemas Agrícolas Tradicionais estivesse concomitantemente associada a técnicas de Aprendizado de Máquina e sensoriamento remoto, operando sobre título, resumo e palavras-chave. A string aplicada no Scopus foi reproduzida integralmente a seguir, mantendo os limites temporais utilizados para permitir reprodutibilidade.

\begin{verbatim}
TITLE-ABS-KEY(
	(
		"traditional agricultural system*" OR "traditional farming system*" OR
		"traditional agriculture" OR "traditional agroecosystem*" OR
		"sistemas agrícolas tradicionais" OR "sistema agrícola tradicional" OR
		"agricultura tradicional" OR
		"indigenous farming" OR "indigenous agriculture" OR
		"shifting cultivation" OR "slash-and-burn" OR
		"ancestral farming" OR "ancient agriculture"
	)
	AND
	(
		"machine learning" OR "artificial intelligence" OR "deep learning" OR
		"random forest" OR "neural network*" OR "support vector machine*" OR
		"SVM" OR "decision tree*" OR "gradient boosting" OR "CNN" OR
		"long short-term memory" OR "LSTM" OR "yolo" OR
		"remote sensing"
	)
)
AND PUBYEAR > 2009 AND PUBYEAR < 2026
\end{verbatim}

Foram elegíveis registros recuperados pelas strings, dentro do intervalo temporal definido, após remoção de duplicatas. Registros com baixa pontuação final na triagem automatizada foram excluídos na etapa de rastreamento, e os registros remanescentes foram submetidos à verificação manual por avaliadores independentes, com consolidação por consenso. A etapa manual também foi utilizada para confirmar e corrigir a extração de campos bibliométricos e de variáveis analíticas quando necessário.

A extração de dados foi conduzida a partir dos metadados exportados em BibTeX e do conteúdo reportado nos estudos incluídos. Chartaram-se variáveis bibliométricas e descritivas como ano, fonte, afiliações, país e palavras-chave, bem como variáveis metodológicas e de desempenho associadas a algoritmos, produtos e instrumentos, incluindo a presença de validação espacialmente independente, métricas de acurácia reportadas, uso de explicabilidade e indicadores de governança de dados associados a conformidade FAIR.

\subsection[Análises estatísticas]{Análises estatísticas}

O corpus de 244 estudos foi submetido a análises estatísticas descritivas para caracterização bibliométrica. Para as análises inferenciais, selecionou-se um subconjunto de 148 estudos que reportaram métricas de desempenho quantitavas, permitindo a validação dos critérios operacionais da Auditoria Socioecológica Digital sem dissociar síntese narrativa e evidência numérica.

\subsubsection[Análises descritivas e exploratórias do corpus]{Análises descritivas e exploratórias do corpus}

Para a caracterização estrutural do campo científico e mapear as tendências temáticas e os centros de produção de conhecimento, procedeu-se à extração e sistematização de metadados bibliométricos, focando na frequência de palavras-chave e na distribuição geográfica das afiliações institucionais.


A estrutura de interações do domínio foi investigada por meio de Análise de Redes Sociais (ARS) \citep{Csardi2006,Schoch2020}. Construiu-se um grafo não direcionado ponderado utilizando os pacotes \texttt{igraph} e \texttt{ggraph}, onde os nós representam entidades (algoritmos, instrumentos, produtos) onde as arestas indicam a coocorrência nos estudos. 

Para identificar a importância relativa dos elementos na rede, calcularam-se métricas de centralidade (grau e intermediação). A detecção de comunidades temáticas foi realizada via algoritmo de Louvain \citep{Blondel2008}, permitindo a identificação de módulos tecnológicos e padrões de especialização funcional.

A evolução temporal (2010--2025) da produção científica e da adoção de famílias algorítmicas foi analisada mediante testes de tendência, utilizando a correlação de Spearman \citep{Spearman1904} para verificar a monotonicidade das séries temporais.

\subsubsection[Análises inferenciais de validação dos critérios operacionais]{Análises inferenciais de validação dos critérios operacionais}

Para quantificar empiricamente as lacunas metodológicas e fundamentar os critérios operacionais da Auditoria Socioecológica Digital, conduziram-se quatro análises inferenciais complementares. Buscando analisar o impacto da validação espacial no desempenho preditivo, compararam-se modelos com particionamento geograficamente independente ($n=70$) versus aleatório convencional ($n=78$), calculando degradação percentual de desempenho entre validação interna e testes externos.

Diferenças foram avaliadas por Mann-Whitney U \citep{Mann1947}, com tamanho de efeito quantificado pelo $d$ de Cohen \citep{Cohen1988}, interpretado como pequeno ($d=0{,}2$), médio ($d=0{,}5$) ou grande ($d=0{,}8$) conforme \citep{Sawilowsky2009}.

Quanto a Regressão logística \citep{Hosmer2013}, esta foi utilizada para estimar a razão de chances de alta performance ($\text{acurácia} \geq 85\%$) em função da validação espacial, controlando por algoritmo e produto, seguindo \citep{Kuhn2013}.

Para investigar o trade off entre explicabilidade e desempenho algorítmico, analisou-se a relação entre explicabilidade algorítmica (escala ordinal 0 a 10 baseada em \citep{Rudin2019}) e acurácia mediante correlação de Spearman \citep{Spearman1904}. Diferenças em acurácia entre modelos com XAI ($n=20$) e sem XAI ($n=128$) foram avaliadas por teste $t$ de Student \citep{Student1908}, verificando normalidade via Shapiro-Wilk \citep{Shapiro1965}. Overhead computacional foi comparado por Mann-Whitney. Análise de Pareto \citep{Pareto1896,Deb2001} identificou algoritmos ótimos sob função de utilidade ponderada $U = 0{,}4 \times \text{acurácia} + 0{,}4 \times \text{explicabilidade} + 0{,}2 \times (1 - \text{tempo normalizado})$.

Visando avaliar a acurácia reportada nos estudos e detectar potencial viés de publicação, conduziu-se meta-análise de efeitos aleatórios \citep{Borenstein2009} com o pacote \texttt{metafor} \citep{Viechtbauer2010}, transformando acurácias via logit para estabilizar variâncias \citep{Barendregt2013} usando $\text{logit}(p)=\ln[p/(1-p)]$. Estimou-se acurácia pooled por modelo REML \citep{DerSimonian1986} com IC 95\%. Heterogeneidade foi quantificada pela estatística $I^2$ \citep{Higgins2003}. Teste Q de Cochran \citep{Cochran1954} avaliou significância da heterogeneidade com $\alpha=0{,}05$. Meta-regressão \citep{Thompson2002} investigou efeitos de ano de publicação e tamanho amostral. Viés de publicação foi detectado por teste de Egger \citep{Egger1997} e método trim and fill \citep{Duval2000}. Forest plots estratificados foram gerados seguindo \citep{Balduzzi2019}.

Por fim, para avaliar a conformidade com princípios de governança de dados abertos, quantificou-se conformidade FAIR mediante score padronizado (0 a 100 pontos) baseado em 12 indicadores binários de \citep{Wilkinson2016}. Cada indicador contribuiu $100/12 \approx 8{,}33$ pontos. Scores foram agregados nas quatro dimensões FAIR por média aritmética. Análise temporal empregou correlação de Spearman \citep{Spearman1904}. Comparações entre estudos com e sem blockchain usaram Mann-Whitney. Gráficos radar multidimensionais visualizaram perfis FAIR com benchmark da Comissão Europeia (75/100) \citep{EC2018}.

Todas as análises foram implementadas em R \citep{RCoreTeam2024} utilizando os pacotes \texttt{ggplot2} \citep{Wickham2016}, \texttt{metafor} \citep{Viechtbauer2010}, \texttt{effsize} \citep{Torchiano2020} e rotinas customizadas para conformidade FAIR. Empregou-se $\alpha=0{,}05$ como nível de significância, aplicando correção de Bonferroni \citep{Bonferroni1936} quando pertinente.

\subsection[Auditoria Socioecológica Digital como sistema de auditoria inferencial]{Auditoria Socioecológica Digital como sistema de auditoria inferencial}

A partir da análise sistemática do corpus bibliográfico, identificou-se que aplicações de aprendizado de máquina em Sistemas Agrícolas Tradicionais carecem de um framework conceitual que integre capacidades computacionais com requisitos de auditabilidade socioecológica, sobretudo quando o objetivo envolve salvaguarda territorial, integridade de serviços ecossistêmicos e transparência decisória. Para preencher essa lacuna, propõe-se o conceito de Auditoria Socioecológica Digital como sistema de auditoria inferencial derivado empiricamente da revisão.

\section[Resultados e discussão]{Resultados e discussão}\label{sec:resultados}

A análise sistemática do corpus de 244 estudos (2010--2025) indica a emergência de um campo científico em franca expansão, cuja maturidade técnica evoluiu de abordagens exploratórias para arquiteturas de inteligência artificial de alta complexidade. A trajetória temporal das publicações sugere um crescimento exponencial, partindo de uma produção incipiente em 2010 (10 estudos) para um volume robusto no biênio 2024--2025 (65 estudos acumulados), sinalizando a consolidação do monitoramento remoto como paradigma central na gestão de agroecossistemas \citep{Weiss2020,Osco2021} (Fig. \ref{fig:temporal_combined}a).

Ao examinar as tendências tecnológicas de forma decomposta (Fig. \ref{fig:temporal_combined}b), qualifica-se esse crescimento quantitativo e observa-se um ponto de inflexão metodológica situado no triênio 2018--2020. Enquanto a fase inicial foi sustentada por classificadores de aprendizado de máquina convencionais (CML), notadamente \textit{Random Forest} e SVM \citep{Belgiu2016,Mountrakis2011}, cuja eficácia em dados espectrais de média resolução contribuiu para a hegemonia inicial, o período recente (2020--2025) é marcado por crescente adoção de arquiteturas de \textit{Deep Learning} \citep{Osco2021,Weiss2020}. 

\begin{figure}[ht]
\centering
\begin{minipage}{0.48\textwidth}
\centering
\includegraphics[width=\linewidth]{../2-FIGURAS/2-EN/temporal_publicacoes.png}
\textbf{(a)}
\end{minipage}\hfill
\begin{minipage}{0.48\textwidth}
\centering
\includegraphics[width=\linewidth]{../2-FIGURAS/2-EN/temporal_algoritmos.png}
\textbf{(b)}
\end{minipage}
\caption{Dinâmica temporal da pesquisa (2010--2025). (a) Evolução exponencial do volume de publicações. (b) Trajetória de substituição tecnológica, evidenciando a ascensão do Deep Learning face à saturação dos métodos convencionais.}
\label{fig:temporal_combined}
\end{figure}


Esta transição é consistente com uma resposta adaptativa à complexidade crescente dos dados de observação da Terra, cenário no qual a saturação de desempenho dos métodos clássicos frente a séries temporais densas e imagens de altíssima resolução tem impulsionado a adoção de Redes Neurais Convolucionais (CNNs) e arquiteturas híbridas \citep{Weiss2020,Osco2021}. Tais modelos têm mostrado capacidade de capturar a heterogeneidade espacial e temporal inerente aos SAT, reduzindo limitações de engenharia de características manual dos modelos antecessores \citep{Osco2021}.

Identificou-se na triagem um núcleo de estudos de Alta aderência (Score $\geq$ 15), que são consistentes com o estado da arte na integração entre inteligência computacional e sistemas tradicionais. Destaca-se o trabalho de \cite{Li2025}, que propõe sistemas unificados de AIoT para controle de pragas, e a pesquisa de \cite{Tripathi2025}, que desenvolve arquiteturas híbridas de Deep Learning para diagnóstico fitopatológico.

O resumo dos principais estudos de alta relevância (Tabela \ref{tab:top7}) evidencia a diversidade de aplicações desde a Amazônia até o Sudeste Asiático.

\begin{table}[h]
\caption{Estudos de Alta aderência sobre em ML aplicado a SAT (2024-2025)}
\label{tab:top7}
\begin{tabular}{@{}llp{8cm}@{}}
\toprule
Ano & Ref & Título e Contribuição \\
\midrule
2025 & \cite{Li2025} & \textit{Unified Pest Prevention...}: Sistema AIoT integrado para controle autônomo e sustentável. \\
2025 & \cite{Tripathi2025} & \textit{Hybrid deep learning system...}: Arquitetura híbrida para classificação de doenças com alta acurácia. \\
2025 & \cite{Ghilardi2025} & \textit{NDVI time series...}: Uso de séries temporais Landsat para monitoramento de degradação em Madagascar. \\
2025 & \cite{Spyrou2025} & \textit{XR-Based Digital Twins...}: Gêmeos digitais imersivos para educação e gestão em agricultura. \\
2025 & \cite{Persson2025} & \textit{Agrarian transitions...}: Análise de sensoriamento remoto sobre transições agrárias no Vietnã/Laos. \\
2024 & \cite{Trehard2024} & \textit{Impact of mobility...}: Correlação entre mobilidade humana e vetores ambientais na Amazônia. \\
2024 & \cite{Li2024} & \textit{Normalized Difference Red-NIR-SWIR...}: Novo índice espectral para mapeamento de agricultura itinerante. \\
\bottomrule
\end{tabular}
\end{table}

\subsection[Panorama das aplicações de aprendizado de máquina]{Panorama das aplicações de aprendizado de máquina em Sistemas Agrícolas}

Observa-se na topologia da rede de conhecimento (Figura \ref{fig:network_completa}), caracterizada por uma densidade de 0,3455 e um diâmetro de 4, um campo científico densamente conectado, onde nós centrais estruturam o fluxo de informação entre domínios distintos. A visualização global destaca a interconectividade entre métodos computacionais e objetos de estudo agrícolas.

\begin{figure}[h]
\centering
\includegraphics[width=1\textwidth]{../2-FIGURAS/2-EN/network_completa.png}
\caption{Topologia da rede de conhecimento e coocorrência de termos.}
\label{fig:network_completa}
\end{figure}

Ao aprofundar a análise estrutural, a detecção de comunidades (Figura \ref{fig:network_communities}) sugere clusters temáticos bem definidos. A metrificação de centralidade aponta a proeminência do cluster \textit{Americas}, refletindo a intensa aplicação de geotecnologias em paisagens agrícolas neotropicais, fortemente acoplado a nós tecnológicos como \texttt{NeuralNetwork} e \texttt{DeepLearning}. Esta configuração sugere uma transição metodológica onde, diferentemente da primeira década (2010--2020) focada na validação de sensores espectrais (NIR/FTIR), o período recente (2021--2025) é marcado por maior participação de arquiteturas de aprendizado profundo aplicadas à fusão de dados multi-sensores \citep{Liakos2018,Weiss2020,Osco2021}.

\begin{figure}[h]
\centering
\IfFileExists{../2-FIGURAS/2-EN/louvain_modules_detailed.png}{%
\includegraphics[width=1\textwidth]{../2-FIGURAS/2-EN/louvain_modules_detailed.png}%
}{\fbox{Figura ausente: louvain\_modules\_detailed.png}}
\caption{Módulos tecnológicos identificados na rede de coocorrência. \textit{Nota:} comunidades detectadas pelo algoritmo de Louvain; rede com 20 nós e 58 arestas; densidade = 0.305; coeficiente de clusterização = 0.595.}
\label{fig:network_communities}
\end{figure}

Essa transição é corroborada pela frequência de palavras-chave (Tabela \ref{tab:keywords}), onde \textit{Remote Sensing} e \textit{Deep Learning} emergem como termos dominantes, indicando a ferramenta preferencial e a abordagem analítica predominante, respectivamente.

\begin{table}[h]
\centering
\caption{Principais palavras-chave associadas aos estudos de ML e SAT (2010--2025)}
\label{tab:keywords}
\begin{tabular}{@{}lc@{}}
\toprule
Palavra-chave & Frequência \\
\midrule
Remote Sensing & 98 \\
Shifting Cultivation & 49 \\
Deforestation & 31 \\
Land Use & 24 \\
Landsat & 22 \\
Deep Learning & 20 \\
Machine Learning & 18 \\
Land Use Change & 17 \\
GIS & 15 \\
Artificial Intelligence & 13 \\
\bottomrule
\end{tabular}
\end{table}


\subsection[Análise de Correspondência Múltipla]{Análise de Correspondência Múltipla (MCA)}

Para compreender as associações latentes entre as categorias metodológicas e geográficas, a Análise de Correspondência Múltipla (MCA) \citep{Greenacre2017,Abdi2014} projetou as variáveis no espaço fatorial, com destaque para a dimensão temporal da análise (Figura \ref{fig:mca_temporal}), que elucida a trajetória evolutiva do campo. O biplot revela a estrutura de variância conjunta, onde a proximidade entre pontos indica uma forte associação estatística \citep{Greenacre2017}. Observa-se um deslocamento claro do centroide 2010--2015, associado a técnicas convencionais e monitoramento local, em direção ao centroide 2020--2025, fortemente correlacionado com Big Data, Deep Learning e escalas globais de análise.

\begin{figure}[h]
\centering
\includegraphics[width=1\textwidth]{../2-FIGURAS/2-EN/mca_biplot_temporal_completo.png}
\caption{Biplot Temporal: Evolução das associações temáticas (2010--2025).}
\label{fig:mca_temporal}
\end{figure}

Nota-se um agrupamento distinto de técnicas de aprendizado profundo associadas a aplicações de sensoriamento remoto, contrastando com métodos estatísticos clássicos vinculados a dados de levantamento de campo \citep{Osco2021,Weiss2020}.

Ao detalhar as especificidades de cada domínio mediante projeção segregada das categorias, nota-se que certas culturas agrícolas tradicionais, como o cultivo itinerante e sistemas agroflorestais, ocupam nichos específicos no espaço multivariado, exigindo combinações particulares de sensores e algoritmos para sua caracterização adequada \citep{Weiss2020,Liakos2018,Li2024}. 

Além disso, o exame das tabelas de contingência quantifica as intensidades dessas associações, destacando, por exemplo, o uso preferencial de Random Forest para classificação de cobertura do solo em cultivos anuais \citep{Belgiu2016}, enquanto Redes Neurais Convolucionais (CNNs) predominam na detecção de doenças em culturas perenes \citep{Tripathi2025,Liakos2018}.

O exame detalhado dos 18 estudos com alta aderência, que apresentaram maior impacto e rigor metodológico, elucida as fronteiras dessa transição. Trabalhos seminais recentes, como o de \cite{Li2024}, demonstram a capacidade de novos índices espectrais perante o desafio de monitoramento.

\subsection[Confronto com a Literatura e Implicações para a Governança]{Confronto com a Literatura e Implicações para a Governança}

Ao contrastar estes resultados com revisões anteriores, nota-se uma mudança de paradigma. Enquanto estudos da década passada focavam na viabilidade técnica de sensores isolados \citep{Liakos2018}, a fronteira atual, representada pelos trabalhos de 2024 e 2025, deslocou-se para a integração de sistemas complexos. A emergência de soluções baseadas em AIoT, como proposto por \cite{Li2025}, sugere que a auditoria ambiental não se limitará mais à observação passiva via satélite, mas incorporará redes de sensores in situ para validação em tempo real.

Contudo, a sofisticação tecnológica observada expõe uma fragilidade na governança de dados. A baixa conformidade FAIR (Figura \ref{fig:fair_radar}), onde apenas 12,8\% dos estudos atingem scores adequados, contrasta com a exigência de transparência dos mercados de carbono e certificação de origem.

\begin{figure}[htbp]
\centering
\begin{minipage}[t]{0.49\textwidth}
\centering
\includegraphics[width=\textwidth]{../2-FIGURAS/2-EN/fair_radar_2.png}
\par\small (a)
\end{minipage}
\hfill
\begin{minipage}[t]{0.49\textwidth}
\centering
\IfFileExists{../2-FIGURAS/2-EN/fair_indicadores.png}{%
\includegraphics[width=\textwidth]{../2-FIGURAS/2-EN/fair_indicadores.png}%
}{\fbox{Figura ausente: fair\_indicadores.png}}
\par\small (b)
\end{minipage}
\caption{Conformidade com princípios FAIR de governança de dados: (a) radar FAIR; (b) conformidade por indicador.}
\label{fig:fair_radar}
\end{figure}

Sem a disponibilização aberta de códigos e dados de treinamento, a reprodutibilidade alegada em muitos estudos fica limitada, criando o que \cite{Shakeripour20241257} identificam como um falso otimismo na maturidade da IA agrícola. 

A dependência de caixas-pretas algorítmicas, embora eficiente em acurácia, conforme sugerido na Figura \ref{fig:meta_algoritmo} (com médias superiores a 90\%), colide com exigências emergentes de transparência e justificabilidade em sistemas de decisão automatizada \citep{Rudin2019}. Essa colisão pode tornar vulnerável qualquer sistema de certificação que se baseie puramente em métricas de desempenho opacas, considerando que análises de trade-off sugerem manutenção de acurácia mesmo em modelos explicáveis, com custo de interpretabilidade potencialmente marginal \citep{Rudin2019}.

\begin{figure}[htbp]
\centering
\begin{minipage}[t]{0.49\textwidth}
\centering
\includegraphics[width=\textwidth]{../2-FIGURAS/2-EN/meta_analise_algoritmos.png}
\par\small (a)
\end{minipage}
\hfill
\begin{minipage}[t]{0.49\textwidth}
\centering
\includegraphics[width=\textwidth]{../2-FIGURAS/2-EN/meta_regressao_ano.png}
\par\small (b)
\end{minipage}
\caption{A síntese meta-analítica das acurácias por algoritmo é acompanhada pela meta-regressão que examina a tendência temporal de desempenho reportado ao longo do período de publicação.}
\label{fig:meta_algoritmo}
\end{figure}


A validade ecológica dos modelos também enfrenta o desafio da falácia da escala. A eficácia reportada por \cite{Li2024} na detecção de clareiras de agricultura itinerante com novos índices espectrais (Red-NIR-SWIR) é promissora, mas sua aplicação em biomas distintos, a exemplo das florestas de Miombo analisadas por \cite{Andrews2024} ou das estepes da Mongólia estudadas por \cite{Prudnikova2024496}, carece de testes de robustez. A degradação de desempenho observada em testes externos sugere que modelos treinados em nichos ecológicos específicos podem falhar ao tentar generalizar padrões para paisagens biogeograficamente distintas.

Isso é consistente com a tese de \cite{Ghilardi2025} de que a auditoria de habitats críticos exige calibração local intensiva, questionando a ideia de algoritmos universais para monitoramento de biodiversidade sem auditabilidade.

\subsection[Limitações]{Limitações}

Algumas limitações inerentes ao delineamento devem ser consideradas ao interpretar os achados. A cobertura das fontes de informação foi restrita a Scopus e Web of Science, o que pode subrepresentar literatura cinzenta, bases regionais. A triagem automatizada, embora apresente precisão de 94,2\%, pode induzir exclusões por sensibilidade limitada em registros com resumos incompletos, terminologia não padronizada ou metadados escassos. Adicionalmente, a síntese quantitativa baseou-se em um subconjunto de 148 estudos com métricas de desempenho reportadas, o que introduz dependência de reporte e aumenta a suscetibilidade a viés de publicação e heterogeneidade entre delineamentos, ainda que tais efeitos tenham sido avaliados por estatísticas e testes apropriados na meta-análise.

Também não se procedeu ao charting sistemático das fontes de financiamento reportadas nos estudos incluídos, de modo que a interpretação sobre drivers institucionais e econômicos da produção científica permanece fora do escopo desta síntese.

\section[Conclusoes]{Conclusões}

Esta pesquisa sistematiza a compreensão de Sistemas Agrícolas Tradicionais como sistemas socioecológicos acoplados, cuja tipicidade emerge de interações não lineares entre variáveis edafoclimáticas e práticas culturais. Os resultados são consistentes com a emergência de um paradigma de monitoramento digital em franca expansão no período analisado (2010--2025), ancorado na integração de sensores remotos e arquiteturas de aprendizado profundo (Deep Learning). A topologia da rede de conhecimento, com densidade de 0,3455 e forte centralidade no cluster \textit{Americas}, sugere um ecossistema de pesquisa coeso, porém regionalmente concentrado, indicando que a validação de algoritmos ainda depende fortemente de contextos neotropicais específicos.

Diferentemente de revisões anteriores que apontavam fragmentação metodológica, os dados atuais indicam uma convergência tecnológica em torno de soluções híbridas e explicáveis. O exame dos 18 estudos de Alta aderência sugere que parte da fronteira do conhecimento tem avançado além da simples classificação de cobertura do solo, aproximando-se de gêmeos digitais com potencial de inferir fitopatologias, dinâmica de carbono e mobilidade humana em condições operacionais reportadas. Contudo, a transposição dessas evidências para decisões de salvaguarda enfrenta o desafio da robustez externa, pois modelos treinados em contextos de alta disponibilidade de dados, carecem de validação sistemática em paisagens de agricultura itinerante na Ásia ou África, limitando sua escalabilidade regulatória.

Em suma, a auditabilidade socioecológica demandada pela governança contemporânea não se limita à acurácia algorítmica, que pode atingir níveis de Alta aderência, mas também envolve a construção de infraestruturas de dados FAIR e a adoção de protocolos de validação que favoreçam a soberania epistêmica das comunidades locais frente à variabilidade climática e territorial. O futuro da pesquisa em SAT tende a convergir para a transição de classificadores estáticos para sistemas adaptativos e auditáveis, com potencial para converter terabytes de dados orbitais em evidências verificáveis de integridade socioecológica.

\section*{Agradecimentos}

Os autores agradecem à Universidade Federal de Sergipe (UFS), à Universidade Estadual de Feira de Santana (UEFS) e ao Instituto Federal de Sergipe (IFS) pelo apoio institucional e infraestrutural que possibilitou esta pesquisa.

\section*{Financiamento}

Esta pesquisa não recebeu financiamento específico de agências de fomento do setor público, comercial ou sem fins lucrativos.

\section*{Conflitos de interesse}

Os autores declaram não haver conflitos de interesse.

\section*{Declaração de disponibilidade de dados}

O conjunto de dados completo que apoia os resultados deste estudo, incluindo o corpus bibliográfico, os scripts de análise e os resultados intermediários, está disponível publicamente no repositório Open Science Framework (OSF) sob DOI \url{https://doi.org/10.17605/OSF.IO/J7STC}.

\bibliography{referencias}

\end{document}
