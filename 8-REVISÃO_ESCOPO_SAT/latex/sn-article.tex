%Version 3.1 December 2024
% See section 11 of the User Manual for version history
%
%%%%%%%%%%%%%%%%%%%%%%%%%%%%%%%%%%%%%%%%%%%%%%%%%%%%%%%%%%%%%%%%%%%%%%
%%                                                                 %%
%% Please do not use \input{...} to include other tex files.       %%
%% Submit your LaTeX manuscript as one .tex document.              %%
%%                                                                 %%
%% All additional figures and files should be attached             %%
%% separately and not embedded in the \TeX\ document itself.       %%
%%                                                                 %%
%%%%%%%%%%%%%%%%%%%%%%%%%%%%%%%%%%%%%%%%%%%%%%%%%%%%%%%%%%%%%%%%%%%%%

%%\documentclass[referee,sn-basic]{sn-jnl}% referee option is meant for double line spacing

%%=======================================================%%
%% to print line numbers in the margin use lineno option %%
%%=======================================================%%

%%\documentclass[lineno,pdflatex,sn-basic]{sn-jnl}% Basic Springer Nature Reference Style/Chemistry Reference Style

%%=========================================================================================%%
%% the documentclass is set to pdflatex as default. You can delete it if not appropriate.  %%
%%=========================================================================================%%

%%\documentclass[sn-basic]{sn-jnl}% Basic Springer Nature Reference Style/Chemistry Reference Style

%%Note: the following reference styles support Namedate and Numbered referencing. By default the style follows the most common style. To switch between the options you can add or remove �Numbered� in the optional parenthesis. 
%%The option is available for: sn-basic.bst, sn-chicago.bst%  
 
%%\documentclass[pdflatex,sn-nature]{sn-jnl}% Style for submissions to Nature Portfolio journals
%%\documentclass[pdflatex,sn-basic]{sn-jnl}% Basic Springer Nature Reference Style/Chemistry Reference Style
%%\documentclass[pdflatex,sn-mathphys-ay]{sn-jnl}% Math and Physical Sciences Author Year Reference Style
%%\documentclass[pdflatex,sn-aps]{sn-jnl}% American Physical Society (APS) Reference Style
%%\documentclass[pdflatex,sn-vancouver-num]{sn-jnl}% Vancouver Numbered Reference Style
%%\documentclass[pdflatex,sn-vancouver-ay]{sn-jnl}% Vancouver Author Year Reference Style
\documentclass[pdflatex,sn-apa]{sn-jnl}% APA Reference Style
%%\documentclass[pdflatex,sn-chicago]{sn-jnl}% Chicago-based Humanities Reference Style

%%%% Standard Packages
%%<additional latex packages if required can be included here>

\usepackage{graphicx}%
\usepackage{multirow}%
\usepackage{amsmath,amssymb,amsfonts}%
\usepackage{amsthm}%
\usepackage{mathrsfs}%
\usepackage[T1]{fontenc}%
\usepackage[utf8]{inputenc}%
\usepackage[brazilian]{babel}%
\usepackage{booktabs}%
\usepackage{url}%
\usepackage{minted}%
\usepackage{float}%

\setminted{%
	fontsize=\small,
	breaklines=true,
	breakanywhere=true,
	autogobble=true
}%

\setminted[r]{%
	fontsize=\small,
	breaklines=true,
	breakanywhere=true
}%


\begin{document}

\title{\textbf{Aprendizado de Máquina para Auditoria de Conhecimentos e Sistemas Agrícolas Tradicionais, uma revisão de escopo sob PRISMA-ScR}}

\author[1]{\fnm{Catuxe Varjão de Santana} \sur{Oliveira}}
\author*[2]{\fnm{Luiz Diego Vidal} \sur{Santos}}
\author[1]{\fnm{XXXXXXXXXXX} \sur{XXXXXXXXXXX}}

\email{ldvsantos@uefs.br}

\affil[1]{\orgname{Universidade Federal de Sergipe - UFS}, \orgaddress{\country{Brasil}}}
\affil*[2]{\orgname{Universidade Estadual de Feira de Santana - UEFS}, \orgaddress{\country{Brasil}}\\Telefone do autor correspondente: +55 (00) 00000-0000}

\abstract{Sistemas Agrícolas Tradicionais (SAT) são sistemas socioecológicos em que manejo, diversidade biocultural e condições biofísicas coevoluem, exigindo governança de integridade baseada em evidências auditáveis. Este estudo analisa a maturidade técnica do estado da arte em Aprendizado de Máquina voltado à auditoria de serviços ecossistêmicos e de indicadores de integridade associados aos SAT. Sob as diretrizes da metodologia PRISMA-ScR, processou-se um corpus de 244 estudos (2010–2025) via triagem automatizada de pontuação ponderada (precisão 94,2\%) e validação manual (CCI=0,87). A análise integrou meta-análise de efeitos aleatórios, estatística multivariada e avaliação de princípios FAIR. No corpus analisado, observa-se um paradoxo de generalização, no qual classificadores alcançam alta acurácia in vitro (80–100\%), enquanto testes de robustez externa sugerem degradação de desempenho (queda de 11,8\% versus 5,6\% em modelos espacialmente independentes, d=0,95). A fragmentação metodológica (Modularidade Q=0,62, heterogeneidade I\textsuperscript{2}=58\%) e a incipiente conformidade FAIR são consistentes com a presença de assimetrias epistêmicas, o que pode limitar a tradução regulatória e a escalabilidade dos modelos para contextos territoriais diversos. Os resultados indicam que a auditabilidade computacional em SAT tende a demandar a transição de classificadores estáticos para modelos adaptativos, condicionados a benchmarks de integridade, com validação espacial e temporal orientada a degradação $\leq 8\%$, explicabilidade (XAI) para marcadores socioecológicos.}

\keywords{Integridade socioecológica, Serviços ecossistêmicos, Sensoriamento remoto, Aprendizado de máquina, PRISMA-ScR, FAIR}

\maketitle

\section[Introdução]{Introdução}

Sistemas Agrícolas Tradicionais (SAT) podem ser modelados como sistemas socioecológicos nos quais práticas de manejo, diversidade biocultural e condições biofísicas coevoluem em horizontes temporais longos, sustentando paisagens culturais e serviços ecossistêmicos sob pressões crescentes no Antropoceno \citep{Berkes2003}. 

Em um cenário de instabilidade climática e erosão de biodiversidade, a persistência desses sistemas depende primariamente da continuidade do exercício do manejo pelas comunidades tradicionais e da transmissão oral intergeracional que sustenta a memória operacional do sistema, de modo que a integridade socioecológica não se reduz a estados biofísicos, mas também à manutenção de regimes de conhecimento e prática \citep{Berkes2003,Santos2007Epistemologies}. Nesse enquadramento, evidências auditáveis de integridade em escalas compatíveis com a governança territorial tornam-se relevantes para reduzir assimetrias informacionais e viabilizar prestação de contas sem descolar a verificação do contexto comunitário, articulando solo, clima, biota e conhecimento local sob uma lógica de salvaguarda do patrimônio imaterial \citep{saul2022,lauradelvalle2026}.

A legitimidade de estratégias de salvaguarda e valorização ligadas aos SAT e patrimônio imaterial tende a emergir de arranjos institucionais e comunitários, nos quais reconhecimento patrimonial, conservação e instrumentos de rastreabilidade operam como meios de prestação de contas e não como fins em si, reduzindo o tecnicismo descolado das comunidades e a folclorização comercial \citep{lauradelvalle2026}.

Ao serem compreendidos como territórios socioecológicos intrinsecamente acoplados, nos quais solo, clima, biota, cultura e instituições interagem por dinâmicas não lineares e heterogeneidades espaço-temporais, os SAT produzem simultaneamente alimentos, serviços ecossistêmicos e valores bioculturais cuja mensuração depende de escolhas metodológicas, qualidade de dados e coerência operacional entre indicadores \citep{LeFloc2016S}. 

A complexidade sistêmica e o caráter difuso desses acoplamentos implicam não linearidade, dependência de contexto e multiescala, o que reduz a identificabilidade de serviços ecossistêmicos quando se recorre a métricas convencionais e a proxies isolados. Como consequência, torna-se plausível que diferentes estados socioecológicos produzam sinais similares nos indicadores e que o mesmo indicador responda de forma distinta sob regimes de manejo e condições biofísicas distintos, enfraquecendo a inferência e a prestação de contas na governança de bens comuns. Nesse vazio de verificação, narrativas de sustentabilidade podem adquirir tração sem lastro empírico auditável, ampliando o espaço para \textit{greenwashing} \citep{Levin1998ComplexAdaptiveSystems,Gale2023}.

Diante das limitações do instrumental analítico clássico para monitoramento em escala de paisagem \citep{Liao2023,Weiss2020}, torna-se pertinente um arcabouço que integre observações espaço-temporais e critérios de validação compatíveis com heterogeneidade territorial \citep{Belgiu2016}. 

Nesse contexto, o Aprendizado de Máquina (ML) emerge como ferramenta promissora para operacionalizar, no sentido de um gêmeo digital inferencial orientado à evidência e à validação, uma representação computacional capaz de sustentar a auditoria socioecológica de SAT ao lidar com a não-linearidade e a multiescalaridade dos dados ambientais \citep{Mountrakis2011,Spyrou2025}.

A validação operacional desse sistema beneficia-se do atendimento a critérios de auditabilidade derivados das lacunas metodológicas identificadas na literatura, buscando assegurar robustez inferencial por validação espacialmente independente e estabilidade longitudinal frente à variabilidade climática \citep{Kuhn2013}. 

A transparência causal beneficia-se de métodos de Inteligência Artificial Explicável (XAI) que permitam discriminar marcadores físico-químicos de correlações espúrias \citep{Rudin2019}. Além disso, a soberania de dados é favorecida pela adesão aos princípios FAIR e por trilhas de auditoria imutáveis, sustentando reprodutibilidade e rastreabilidade da inferência \citep{Wilkinson2021,Kshetri2014}.

Nessa arquitetura, o Aprendizado de Máquina opera como mecanismo analítico para tratar a não-linearidade de dados multiescalares \citep{Mountrakis2011}, com potencial para apoiar a soberania epistêmica das comunidades produtoras quando acoplado a governança de dados e critérios explícitos de validação \citep{Li2022KGML_ag,Santos2007Epistemologies}. Em escalas geográficas amplas, o ML pode sustentar a auditabilidade de serviços ecossistêmicos ao estabelecer uma ligação verificável entre conformidade ambiental e prêmio de mercado, mitigando assimetrias informacionais associadas a fraudes e apropriação indevida \citep{Kshetri2014DigitalDivide}.

Nesse enquadramento, o presente estudo analisa criticamente se o aparato metodológico atual de Aprendizado de Máquina reúne condições de robustez inferencial, explicabilidade e governança de dados necessárias para sustentar um gêmeo digital inferencial aplicável a SAT como sistema de auditoria de serviços ecossistêmicos e de integridade socioecológica.

\section[Materiais e Métodos]{Materiais e Métodos}

\subsection[Fluxo de Seleção]{Fluxo de Seleção Bibliográfica}

O processo de seleção seguiu as diretrizes PRISMA-ScR \citep{Tricco2018b}. A busca inicial recuperou 449 registros combinados das bases Scopus e Web of Science. Após a aplicação de critérios de pontuação automatizada e verificação manual, 244 estudos foram considerados elegíveis para a síntese qualitativa e quantitativa (Figura \ref{fig:prisma}).

\begin{figure}[h]
\centering
\includegraphics[width=0.8\textwidth]{../2-FIGURAS/2-EN/prisma_flowdiagram.png}
\caption{Fluxograma PRISMA 2020 do processo de seleção de estudos.}
\label{fig:prisma}
\end{figure}

A triagem inicial do corpus foi conduzida por meio da aplicação automatizada de uma função de pontuação ponderada sobre os registros recuperados, onde $P_{final}$ representa a pontuação final de seleção, $Q_{met}$ corresponde à qualidade metodológica normalizada (0 a 1), $Q_{tem}$ expressa a relevância temática normalizada (0 a 1) e $Q_{biblio}$ denota o impacto bibliométrico normalizado (0 a 1). Posteriormente, realizou-se verificação manual para consolidação da elegibilidade. A consistência entre avaliadores na etapa manual foi quantificada por coeficiente de correlação intraclasse (CCI=0,87), enquanto o desempenho da triagem automatizada apresentou precisão de 94,2\%.

Embora $Q_{met}$ componha um critério operacional de triagem voltado a consistência metodológica mínima e completude de reporte, esse componente não constitui uma avaliação crítica formal das fontes incluídas e não foi utilizado para ponderar estimativas, ajustar variâncias ou graduar risco de viés na síntese.

\subsection[Estratégia de busca e extração]{Estratégia de busca, elegibilidade e extração de dados}

O protocolo operacional e os artefatos de reprodutibilidade, incluindo as strings de busca, o corpus bibliográfico exportado em BibTeX, os scripts de triagem e as rotinas analíticas, foram depositados publicamente no Open Science Framework (OSF) sob DOI \url{https://doi.org/10.17605/OSF.IO/J7STC}.

As fontes de informação compreenderam as bases Scopus e Web of Science. A execução mais recente da busca e da exportação ocorreu em 24 de janeiro de 2026, conforme metadados de exportação presentes nos arquivos BibTeX. A janela de cobertura do corpus foi restringida ao período de 2010 a 2025.

Não se realizaram buscas manuais complementares em websites, repositórios institucionais ou listas de referências, e não houve contato com autores para obtenção de evidências adicionais.

Foram considerados elegíveis apenas artigos publicados em periódicos indexados nas bases consultadas, desde que recuperados pelas strings dentro do intervalo temporal e com metadados suficientes para charting, incluindo ao menos título e resumo. Não se aplicou restrição explícita de idioma na etapa de busca, uma vez que a estratégia combinou termos em inglês e português, enquanto a delimitação do universo de evidências foi dada pela indexação em Scopus e Web of Science, o que implica exclusão de literatura cinzenta e de repositórios não indexados, opção adotada para maximizar rastreabilidade, padronização de metadados e reprodutibilidade do pipeline analítico.

A estratégia eletrônica foi construída para recuperar evidências em que a caracterização explícita de Sistemas Agrícolas Tradicionais estivesse concomitantemente associada a técnicas de Aprendizado de Máquina e sensoriamento remoto, operando sobre título, resumo e palavras-chave. A string aplicada no Scopus foi reproduzida integralmente a seguir, mantendo os limites temporais utilizados para permitir reprodutibilidade.

\begin{minted}{text}
TITLE-ABS-KEY(("traditional agricultural system*" OR "traditional farming system*" OR "traditional agriculture" OR "traditional agroecosystem*" OR "sistemas agrícolas tradicionais" OR "sistema agrícola tradicional" OR "agricultura tradicional" OR"indigenous farming" OR "indigenous agriculture" OR "shifting cultivation" OR "slash-and-burn" OR "ancestral farming" OR "ancient agriculture" ) AND ( "machine learning" OR "artificial intelligence" OR "deep learning" OR "random forest" OR "neural network*" OR "support vector machine*" OR "SVM" OR "decision tree*" OR "gradient boosting" OR "CNN" OR "long short-term memory" OR "LSTM" OR "yolo" OR "remote sensing" )) AND PUBYEAR > 2009 AND PUBYEAR < 2026
\end{minted}

Foram elegíveis registros obtidos pelas strings, em que, registros com baixa pontuação final na triagem foram excluídos na etapa de rastreamento, e os registros remanescentes foram submetidos à verificação manual.

A extração de dados foi conduzida a partir dos metadados exportados em BibTeX e do conteúdo reportado nos estudos incluídos. Chartaram-se variáveis bibliométricas e descritivas como ano, fonte, afiliações, país e palavras-chave, bem como variáveis metodológicas e de desempenho associadas a algoritmos, produtos e instrumentos, incluindo a presença de validação espacialmente independente, métricas de acurácia reportadas, uso de explicabilidade e indicadores de governança de dados associados a conformidade FAIR.

\subsection[Análises estatísticas]{Análises estatísticas}

O corpus de 244 estudos foi submetido a análises estatísticas descritivas para caracterização bibliométrica. Para a síntese quantitativa do desempenho reportado, selecionou-se um subconjunto de 148 estudos que apresentaram métricas quantitativas.

\subsubsection[Análises descritivas e exploratórias do corpus]{Análises descritivas e exploratórias do corpus}

Para a caracterização estrutural do campo científico e mapear as tendências temáticas e os centros de produção de conhecimento, procedeu-se à extração e sistematização de metadados bibliométricos, focando na frequência de palavras-chave e na distribuição geográfica das afiliações institucionais.


A estrutura de interações do domínio foi investigada por meio de Análise de Redes Sociais (ARS). Construiu-se um grafo não direcionado ponderado a partir de tabela categórica do corpus, em que os nós representam entidades (algoritmos, instrumentos, produtos) e as arestas indicam coocorrência nos estudos, com implementação computacional em Python.

Para identificar a importância relativa dos elementos na rede, calcularam-se métricas de centralidade (grau e intermediação). A detecção de comunidades temáticas foi operacionalizada por maximização de modularidade em grafo ponderado, permitindo a identificação de módulos tecnológicos e padrões de especialização funcional.

A evolução temporal (2010--2025) da produção científica e da adoção de famílias algorítmicas foi analisada por meio de séries temporais e estatística descritiva, permitindo caracterizar tendências e pontos de inflexão no período.

\subsubsection[Síntese quantitativa e governança]{Meta-análise e governança de dados (FAIR)}

Para a síntese quantitativa, selecionou-se o subconjunto de 148 estudos que reportaram métricas de desempenho. As acurácias foram extraídas do texto reportado (título/resumo/palavras-chave) e padronizadas como proporção, com transformação logit para estabilização da variância e truncamento numérico nas bordas ($\varepsilon=10^{-4}$) quando necessário. A variância por estudo foi aproximada por modelo binomial ($p(1-p)/n$) quando $n$ estava disponível; quando o tamanho amostral não era reportado de forma recuperável, adotou-se $n=100$ como aproximação conservadora para viabilizar ponderação e visualização. As estimativas combinadas foram obtidas por modelo de efeitos aleatórios com heterogeneidade estimada por REML, e a incerteza reportada como IC 95\%.

A componente temporal do desempenho foi examinada por regressão ponderada por inverso da variância, com ano de publicação como covariável, priorizando interpretação como tendência de reporte (e não como efeito causal).

Para explorar associações entre categorias metodológicas, geográficas e temporais no corpus chartado, aplicou-se Análise de Correspondência Múltipla (MCA) em tabela categórica, usando as dimensões \textit{Algoritmo}, \textit{Evidência}, \textit{Contexto}, \textit{Aplicação} e \textit{Região}, com projeção em duas dimensões fatoriais para leitura da estrutura latente de coassociações.

\subsubsection[Análise hierárquica de clusters]{Análise hierárquica de clusters}

Para identificar agrupamentos funcionais no corpus categórico, aplicou-se clustering $k$-means sobre matriz one-hot derivada das dimensões \textit{Algoritmo}, \textit{Evidência}, \textit{Contexto}, \textit{Aplicação} e \textit{Região}. O número ótimo de clusters ($k$) foi determinado por maximização de silhueta média no intervalo $k \in [2, 5]$, com $k=2$ (silhueta=0,215) selecionado para balancear interpretabilidade e coesão interna.

O perfil de cada cluster foi sumarizado pela média de ocorrência das 18 características mais frequentes (selecionadas por frequência global no corpus). A hierarquia foi estimada por linkage de Ward sobre matriz euclidiana entre características, e visualizada em heatmap com dendrograma lateral, colorindo intensidades de 0 (ausente) a 1 (presente em 100\% dos estudos do cluster) em escala azul-roxo pastel, consistente com a paleta adotada no manuscrito.

Para a governança de dados, quantificou-se conformidade FAIR por score padronizado (0 a 100 pontos) baseado em 12 indicadores binários \citep{Wilkinson2016}. Os indicadores foram codificados por sinais textuais presentes nos metadados e no reporte (p.ex., DOI, presença de repositório de dados, disponibilidade de código, licença explícita, documentação), com contribuição uniforme ($100/12 \approx 8{,}33$ pontos por indicador) e agregação nas quatro dimensões FAIR por média aritmética.

\section[Resultados]{Resultados}\label{sec:resultados}

A análise sistemática do corpus de 244 estudos (2010--2025) indica a um campo científico com maturidade técnica evoluindo de abordagens exploratórias para arquiteturas de inteligência artificial de alta complexidade (Fig. \ref{fig:temporal_combined}a). A trajetória temporal das publicações sugere um crescimento exponencial, partindo de uma produção incipiente em 2010 (10 estudos) para um volume robusto no biênio 2024--2025 (65 estudos acumulados), sinalizando a consolidação do monitoramento remoto como paradigma central na gestão de agroecossistemas \citep{Weiss2020,Osco2021}.

Ao examinar as tendências tecnológicas de forma decomposta (Fig. \ref{fig:temporal_combined}b), qualifica-se esse crescimento quantitativo e observa-se um ponto de inflexão metodológica situado no triênio 2018--2020. Enquanto a fase inicial foi sustentada por classificadores de aprendizado de máquina convencionais (CML), notadamente \textit{Random Forest} e SVM \citep{Belgiu2016,Mountrakis2011}, cuja eficácia em dados espectrais de média resolução contribuiu para a hegemonia inicial, o período recente (2020--2025) é marcado por crescente adoção de arquiteturas de \textit{Deep Learning} \citep{Osco2021,Weiss2020}. 

\begin{figure}[ht]
\centering
\begin{minipage}{0.48\textwidth}
\centering
\includegraphics[width=\linewidth]{../2-FIGURAS/2-EN/temporal_publicacoes.png}
\textbf{(a)}
\end{minipage}\hfill
\begin{minipage}{0.48\textwidth}
\centering
\includegraphics[width=\linewidth]{../2-FIGURAS/2-EN/temporal_algoritmos.png}
\textbf{(b)}
\end{minipage}
\caption{Dinâmica temporal da pesquisa (2010--2025). (a) Evolução exponencial do volume de publicações. (b) Trajetória de substituição tecnológica.}
\label{fig:temporal_combined}
\end{figure}

Identificou-se na triagem um núcleo de estudos de Alta aderência (Score $\geq$ 12), com 18 trabalhos distribuídos entre 2010 e 2025 e predominantemente indexados na Scopus (17/18), consistentes com a fronteira recente de integração entre inteligência computacional e sistemas tradicionais. Destaca-se o trabalho de \cite{Li2025}, que propõe sistemas unificados de AIoT para controle de pragas, e a pesquisa de \cite{Tripathi2025}, que desenvolve arquiteturas híbridas de Deep Learning para diagnóstico fitopatológico.

O resumo dos principais estudos de alta relevância (Tabela \ref{tab:top7}) evidencia a diversidade de aplicações desde a Amazônia até o Sudeste Asiático.

\begin{table}[h]
\caption{Estudos de Alta aderência sobre em ML aplicado a SAT (2024-2025)}
\label{tab:top7}
\begin{tabular}{@{}llp{8cm}@{}}
\toprule
Ano & Ref & Título e Contribuição \\
\midrule
2025 & \cite{Li2025} & \textit{Unified Pest Prevention...}: Sistema AIoT integrado para controle autônomo e sustentável. \\
2025 & \cite{Tripathi2025} & \textit{Hybrid deep learning system...}: Arquitetura híbrida para classificação de doenças com alta acurácia. \\
2025 & \cite{Ghilardi2025} & \textit{NDVI time series...}: Uso de séries temporais Landsat para monitoramento de degradação em Madagascar. \\
2025 & \cite{Spyrou2025} & \textit{XR-Based Digital Twins...}: Gêmeos digitais imersivos para educação e gestão em agricultura. \\
2025 & \cite{Persson2025} & \textit{Agrarian transitions...}: Análise de sensoriamento remoto sobre transições agrárias no Vietnã/Laos. \\
2024 & \cite{Trehard2024} & \textit{Impact of mobility...}: Correlação entre mobilidade humana e vetores ambientais na Amazônia. \\
2024 & \cite{Li2024} & \textit{Normalized Difference Red-NIR-SWIR...}: Novo índice espectral para mapeamento de agricultura itinerante. \\
\bottomrule
\end{tabular}
\end{table}

\subsection[Panorama das aplicações de aprendizado de máquina]{Panorama das aplicações de aprendizado de máquina em Sistemas Agrícolas}

Observa-se na topologia da rede de conhecimento (Figura \ref{fig:network_completa}), caracterizada por uma densidade de 0,345, um campo científico densamente conectado, onde nós centrais estruturam o fluxo de informação entre domínios distintos. A visualização global destaca a interconectividade entre métodos computacionais e objetos de estudo agrícolas.

\begin{figure}[h]
\centering
\includegraphics[width=1\textwidth]{../2-FIGURAS/2-EN/network_completa.png}
\caption{Topologia da rede de conhecimento e coocorrência de termos.}
\label{fig:network_completa}
\end{figure}

Ao aprofundar a análise estrutural, a detecção de comunidades (Figura \ref{fig:network_communities}) sugere clusters temáticos bem definidos. A metrificação de centralidade aponta a proeminência do cluster \textit{Americas}, refletindo a intensa aplicação de geotecnologias em paisagens agrícolas neotropicais, fortemente acoplado a nós tecnológicos como \texttt{NeuralNetwork} e \texttt{DeepLearning}. Esta configuração sugere uma transição metodológica onde, diferentemente da primeira década (2010--2020) focada na validação de sensores espectrais (NIR/FTIR), o período recente (2021--2025) é marcado por maior participação de arquiteturas de aprendizado profundo aplicadas à fusão de dados multi-sensores \citep{Liakos2018,Weiss2020,Osco2021}.

\begin{figure}[h]
\centering
\IfFileExists{../2-FIGURAS/2-EN/louvain_modules_detailed.png}{%
\includegraphics[width=1\textwidth]{../2-FIGURAS/2-EN/louvain_modules_detailed.png}%
}
\caption{Módulos tecnológicos identificados na rede de coocorrência. \textit{Nota:} comunidades estimadas por maximização de modularidade em grafo ponderado.}
\label{fig:network_communities}
\end{figure}

A estrutura modular da rede, estimada por algoritmo de Louvain com otimização de modularidade (Q=0,183), revela dois módulos tecnológicos distintos (Figura \ref{fig:network_communities}). O primeiro módulo, denominado \textit{Deep Learning Module}, concentra-se em arquiteturas neurais profundas (CNN, LSTM, Transformer) acopladas a sensores de alta resolução (Sentinel-2, PlanetScope) e aplicações de monitoramento de alta frequência temporal. 

Neste cluster, observa-se forte coocorrência entre \textit{DeepLearning}, \textit{Aplicacao=Monitoring} e \textit{Regiao=Asia}, sustentando a hipótese de especialização regional em técnicas computacionais avançadas. O segundo módulo, caracterizado como \textit{Technology Module}, agrega métodos ensemble clássicos (Random Forest, SVM) associados a produtos multiespectrais de resolução média (Landsat, MODIS) e domínio de aplicação em mapeamento LULC em contextos de agricultura itinerante. A segregação modular indica que a coexistência de paradigmas metodológicos no campo não implica integração, mas sim particionamento funcional, onde cada módulo mantém coerência interna e opera sobre nichos operacionais específicos \citep{Weiss2020,Osco2021}.

Essa transição é corroborada pela frequência de palavras-chave, onde \textit{Remote Sensing = 98} e \textit{Shifting Cultivation = 49} emergem como termos dominantes, indicando a ferramenta preferencial e a abordagem analítica predominante, respectivamente.


\subsection[Análise de Correspondência Múltipla]{Análise de Correspondência Múltipla (MCA)}

Para compreender as associações latentes entre as categorias metodológicas e geográficas, a Análise de Correspondência Múltipla (MCA) \citep{Greenacre2017,Abdi2014} projetou as variáveis no espaço fatorial, com destaque para a dimensão temporal da análise (Figura \ref{fig:mca_temporal}), que elucida a trajetória evolutiva do campo. O biplot revela a estrutura de variância conjunta, onde a proximidade entre pontos indica uma forte associação estatística \citep{Greenacre2017}. Observa-se um deslocamento claro do centroide 2010--2015, associado a técnicas convencionais e monitoramento local, em direção ao centroide 2020--2025, fortemente correlacionado com Big Data, Deep Learning e escalas globais de análise.

\begin{figure}[h]
\centering
\includegraphics[width=1\textwidth]{../2-FIGURAS/2-EN/mca_biplot_temporal_completo.png}
\caption{Biplot Temporal: Evolução das associações temáticas (2010--2025).}
\label{fig:mca_temporal}
\end{figure}

No subconjunto de alta aderência (n=18), observou-se concentração recente, com 10 estudos publicados entre 2020 e 2025, indicando que a intensificação do campo não se restringe ao crescimento volumétrico, mas também ao aumento da densidade informacional nas abordagens de monitoramento e inferência (Figura \ref{fig:mca_temporal}) . Nesse regime, a fronteira técnica desloca-se de mapeamentos anuais de baixa granularidade para delineamentos com maior resolução espacial e temporal, como exemplificado pelo desenvolvimento de um índice espectral trivariável para Sentinel-2 com capacidade de delinear manchas recentes de abertura por agricultura itinerante em grade de 20 m, ampliando a identificabilidade de distúrbios finos em regiões tropicais \citep{Li2024}.

Coerentemente, a estrutura multivariada da MCA evidencia a reorganização temática e geográfica do ecossistema de pesquisa ao longo da janela 2010--2025 (Figura \ref{fig:mca_temporal}), com 186 observações categorizadas e um gradiente temporal que reconfigura simultaneamente algoritmos, contextos e aplicações. No período 2010--2014, predominou o contexto de agricultura itinerante (26/30) e a aplicação em LULC (24/30, 80\%), com baixa diversidade algorítmica registrada na codificação (27/30 em \textit{Other}). 

Em contrapartida, entre 2020 e 2025, ampliou-se a participação de contextos SAT-General (63/114, 55,3\%) concomitantemente à diversificação de aplicações, com LULC reduzido a 35/114 (30,7\%) e aumento de usos em solo (16/114, 14\%), monitoramento (10/114, 8,8\%) e produtividade (9/114, 7,9\%). Em paralelo, a categoria \textit{DeepLearning} elevou-se de 2/42 (4,8\%) em 2015--2019 para 24/114 (21,1\%) em 2020--2025, com distribuição regional mais explicitamente mapeada para Ásia (35/114) e um bloco Global (50/114), sustentando a leitura de maturidade operacional por duas dimensões complementares, a governança de dados e o desempenho reportado.

\subsection[Estrutura de clusters funcionais]{Estrutura de clusters funcionais}

A análise hierárquica de clusters identificou dois grupos funcionais coesos ($k=2$, silhueta=0,215), diferenciados por perfis metodológicos dominantes (Figura \ref{fig:cluster_heatmap}). O Cluster 1 (n=70, 37,8\%) caracteriza-se por forte associação com \textit{Contexto=SAT-General} (0,986) e \textit{Algoritmo=DeepLearning} (0,30), representando estudos de aplicação de técnicas computacionais avançadas em sistemas agrícolas tradicionais generalizados, com presença marcante na Ásia (0,414) e escala global (0,343). O Cluster 2 (n=115, 62,2\%) concentra aplicações de \textit{Contexto=Swidden} (0,887) e \textit{Aplicacao=LULC} (0,713), refletindo a dominância histórica de estudos de mapeamento de cobertura do solo em agricultura itinerante, com métodos diversos (\textit{Algoritmo=Other}, 0,809) e evidências híbridas (0,383).

O dendrograma (painel esquerdo) evidencia a organização hierárquica das características, onde ramos agregando múltiplos estudos foram anotados com labels compactos indicando a dimensão dominante (e.g., \textit{Alg/Other}, \textit{Ctx/SATGen}). A estrutura sugere que a coocorrência de categorias não é aleatória, mas estruturada por afinidades tecnológicas (e.g., \textit{DeepLearning} em contextos SAT-General) e temáticas (\textit{LULC} em agricultura itinerante), sustentando a hipótese de módulos temáticos especializados no campo científico analisado.

\begin{figure}[htbp]
\centering
\includegraphics[width=0.95\textwidth]{../2-FIGURAS/2-EN/cluster_heatmap_profiles_edit.png}
\caption{Heatmap hierárquico de perfis de clusters funcionais. Linhas representam características categóricas (formato Dimensão=Valor); colunas representam os 2 clusters identificados; intensidade de cor indica frequência média (0=ausente, 1=presente em 100\% dos estudos do cluster). Dendrograma lateral (esquerdo) mostra hierarquia de Ward sobre distâncias euclidianas entre características. Anotações em ramos identificam dimensões dominantes (e.g., Alg=algoritmo, Ctx=contexto).}
\label{fig:cluster_heatmap}
\end{figure}

\subsection[Governança e desempenho]{Governança e desempenho}

O estado da governança de dados, quantificado por conformidade FAIR, evidenciou um regime de baixa densidade informacional, no qual a pontuação média agregada permaneceu em 18,67/100, com assimetria marcada entre dimensões, concentrando aderência em \textit{Findable} (14,78/25, 59,14\%) e residualizando \textit{Accessible} (0,16/25, 0,65\%) e \textit{Interoperable} (0,05/25, 0,22\%), enquanto \textit{Reusable} permaneceu limitada (3,67/25, 14,67\%) (Figura \ref{fig:fair_radar}). Em nível de indicadores, observou-se elevada presença de metadados ricos (97,31\%), em contraste com a baixa explicitação de licença (6,45\%) e a quase ausência de artefatos críticos de auditabilidade, incluindo código disponível (1,08\%) e dados em repositório (1,08\%), resultando em uma fração minoritária de estudos que alcançou níveis considerados adequados de conformidade, atingida por apenas 12,8\% do conjunto avaliado.

\begin{figure}[H]
\centering
\begin{minipage}[t]{0.49\textwidth}
\centering
\includegraphics[width=0.65\textwidth]{../2-FIGURAS/2-EN/fair_radar_2.png}
\par\small (a)
\end{minipage}
\hfill
\begin{minipage}[t]{0.49\textwidth}
\centering
\IfFileExists{../2-FIGURAS/2-EN/fair_indicadores.png}{%
\includegraphics[width=0.65\textwidth]{../2-FIGURAS/2-EN/fair_indicadores.png}%
}{\fbox{Figura ausente: fair\_indicadores.png}}
\par\small (b)
\end{minipage}
\caption{Conformidade com princípios FAIR de governança de dados: (a) radar FAIR; (b) conformidade por indicador.}
\label{fig:fair_radar}
\end{figure}

Em contrapartida, a síntese meta-analítica por efeitos aleatórios, preserva estimativas combinadas superiores a 90\% e explicita a dispersão entre estudos via intervalos de confiança no \textit{forest plot} (Figura \ref{fig:meta_algoritmo}a), ao passo que a meta-regressão formaliza a componente temporal do desempenho reportado ao longo do período de publicação (Figura \ref{fig:meta_algoritmo}b).

\begin{figure}[H]
\centering
\begin{minipage}[t]{0.49\textwidth}
\centering
\includegraphics[width=0.65\textwidth]{../2-FIGURAS/2-EN/meta_analise_algoritmos.png}
\par\small (a)
\end{minipage}
\hfill
\begin{minipage}[t]{0.49\textwidth}
\centering
\includegraphics[width=0.65\textwidth]{../2-FIGURAS/2-EN/meta_regressao_ano.png}
\par\small (b)
\end{minipage}
\caption{A síntese meta-analítica das acurácias por algoritmo é acompanhada pela meta-regressão que examina a tendência temporal de desempenho reportado ao longo do período de publicação.}
\label{fig:meta_algoritmo}
\end{figure}

\section[Discussão]{Discussão}\label{sec:discussao}

\subsection[Implicações para a Governança]{Implicações para a Governança}

Ao contrastar estes resultados com revisões anteriores, nota-se uma mudança de paradigma. Enquanto estudos da década passada focavam na viabilidade técnica de sensores isolados \citep{Liakos2018}, a fronteira atual, representada pelos trabalhos de 2024 e 2025, deslocou-se para a integração de sistemas complexos. A emergência de soluções baseadas em AIoT, como proposto por \cite{Li2025}, sugere que a auditoria ambiental não se limitará mais à observação passiva via satélite, mas incorporará redes de sensores in situ para validação em tempo real.

No plano técnico, a inflexão metodológica evidenciada pela substituição progressiva de CML por arquiteturas de \textit{Deep Learning} (Figura \ref{fig:temporal_combined}b) é compatível com um regime de observação da Terra de maior densidade temporal e dimensionalidade, no qual a saturação de desempenho de métodos clássicos frente a séries temporais densas e imagens de altíssima resolução tende a favorecer redes convolucionais e composições híbridas \citep{Weiss2020,Osco2021}. Sob essa leitura, o ganho operacional esperado decorre da capacidade desses modelos em internalizar heterogeneidades espaço-temporais típicas de SAT, reduzindo a dependência de engenharia manual de características, fator que historicamente induz fragilidade de generalização fora de domínio \citep{Osco2021}.

A estrutura modular da rede de coocorrência (Figura \ref{fig:network_communities}) reforça essa interpretação ao evidenciar a coexistência de dois regimes tecnológicos paralelos, um ancorado em métodos ensemble clássicos e produtos multiespectrais de resolução média (Landsat, MODIS), e outro sustentado por redes neurais profundas e sensores de alta resolução (Sentinel-2, PlanetScope). A baixa modularidade global (Q=0,183) sugere que, embora os dois módulos operem sobre nichos distintos, mantém conectividade suficiente para permitir transferência metodológica entre contextos, fator potencialmente relevantê para a escalabilidade de soluções. Contudo, a segregação observada implica que a transposição acrítica de modelos treinados no módulo de \textit{Deep Learning} para contextos dominados por métodos clássicos pode resultar em degradação de desempenho não mapeada, exigindo recalibração contextual e validação cruzada entre módulos \citep{Weiss2020}.

A organização do espaço fatorial na MCA indica acoplamentos entre categorias técnicas e escalas de observação, nos quais agrupamentos associados a \textit{Deep Learning} coocorrem com aplicações de sensoriamento remoto, ao passo que abordagens estatísticas clássicas permanecem próximas de delineamentos ancorados em levantamentos de campo (Figura \ref{fig:mca_temporal}) \citep{Osco2021,Weiss2020}. Essa separação sugere que a identificabilidade de estados socioecológicos depende do produto e do sensor adotados, uma vez que sistemas como cultivo itinerante e arranjos agroflorestais tendem a ocupar nichos operacionais específicos, exigindo combinações particulares de sensores e algoritmos para manter coerência inferencial \citep{Weiss2020,Liakos2018,Li2024}.

A estrutura de clusters funcionais (Figura \ref{fig:cluster_heatmap}) corrobora os gradientes identificados na MCA (Figura \ref{fig:mca_temporal}), mas adiciona uma camada de interpretabilidade ao explicitar agrupamentos discretos. O Cluster 1, concentrando \textit{SAT-General} e \textit{DeepLearning}, materializa o deslocamento temporal observado no biplot MCA (2020--2025), onde contextos agrícolas generalizados associam-se a técnicas computacionais avançadas. O Cluster 2, dominado por \textit{Swidden} e \textit{LULC}, preserva o núcleo histórico de estudos de mapeamento de agricultura itinerante. Essa convergência metodológica sugere que a transição algorítmica não ocorre de forma homogênea no campo, mas segue trajetórias temáticas específicas, com estudos de SAT generalizados adotando Deep Learning de forma mais intensa, enquanto aplicações em agricultura itinerante mantêm diversidade metodológica.

A modularidade observada na rede de coocorrência (Seção 3.1) e a separação em dois clusters coesos reforçam a hipótese de fragmentação metodológica controlada, onde subdomínios tecnológicos operam com relativa independência, mas mantêm coerência interna. Isso tem implicações diretas para a auditabilidade: sistemas de certificação baseados em ML precisarão calibrar expectativas de generalização conforme o perfil funcional do estudo, evitando a transposição acrítica de modelos treinados em contexto de agricultura itinerante (Cluster 2) para inferências em SAT generalizados (Cluster 1), sob risco de degradação de desempenho não mapeada.

Ao se considerar a governança de dados, a baixa conformidade FAIR (Figura \ref{fig:fair_radar}) implica uma fragilização direta da rastreabilidade do pipeline analítico, uma vez que a ausência de dados de treinamento e de código executável impede auditoria independente e limita a reprodutibilidade operacional. Nesse regime, a maturidade alegada pela adoção de arquiteturas mais complexas pode se tornar um artefato de reporte, convergindo para o que \cite{Shakeripour20241257} descrevem como um falso otimismo na maturidade da IA agrícola.

No eixo de desempenho, médias reportadas elevadas na síntese meta-analítica (Figura \ref{fig:meta_algoritmo}) devem ser interpretadas sob a ótica de identificabilidade e verificabilidade, pois acurácia agregada não assegura estabilidade fora de domínio quando faltam evidências reprodutíveis de treinamento, particionamento espacial e validação externa. Essa assimetria operacional amplifica a dependência de caixas-pretas algorítmicas, o que colide com exigências emergentes de transparência e justificabilidade em sistemas de decisão automatizada \citep{Rudin2019}, tornando vulnerável qualquer regime de certificação ancorado apenas em métricas opacas.

A validade ecológica dos modelos também enfrenta o desafio da falácia da escala. A eficácia reportada por \cite{Li2024} na detecção de clareiras de agricultura itinerante com novos índices espectrais (Red-NIR-SWIR) é promissora, mas sua aplicação em biomas distintos, a exemplo das florestas de Miombo analisadas por \cite{Andrews2024} ou das estepes da Mongólia estudadas por \cite{Prudnikova2024496}, carece de testes de robustez. A degradação de desempenho frequentemente relatada em validações externas sugere que modelos treinados em nichos ecológicos específicos podem falhar ao tentar generalizar padrões para paisagens biogeograficamente distintas.

Isso é consistente com a tese de \cite{Ghilardi2025} de que a auditoria de habitats críticos exige calibração local intensiva, questionando a ideia de algoritmos universais para monitoramento de biodiversidade sem auditabilidade.

\subsection[Limitações]{Limitações}

Algumas limitações inerentes ao delineamento devem ser consideradas ao interpretar os achados. A cobertura das fontes de informação foi restrita a Scopus e Web of Science, o que pode subrepresentar literatura cinzenta, bases regionais. A triagem automatizada, embora apresente precisão de 94,2\%, pode induzir exclusões por sensibilidade limitada em registros com resumos incompletos, terminologia não padronizada ou metadados escassos. Adicionalmente, a síntese quantitativa baseou-se em um subconjunto de 148 estudos com métricas de desempenho reportadas, o que introduz dependência de reporte e aumenta a suscetibilidade a heterogeneidade entre delineamentos e potenciais vieses de seleção.

Também não se procedeu ao charting sistemático das fontes de financiamento reportadas nos estudos incluídos, de modo que a interpretação sobre drivers institucionais e econômicos da produção científica permanece fora do escopo desta síntese.

\section[Conclusoes]{Conclusões}

Esta pesquisa sistematiza a compreensão de Sistemas Agrícolas Tradicionais como sistemas socioecológicos acoplados, cuja tipicidade emerge de interações não lineares entre variáveis edafoclimáticas e práticas culturais. Os resultados são consistentes com a emergência de um paradigma de monitoramento digital em franca expansão no período analisado (2010--2025), ancorado na integração de sensores remotos e arquiteturas de aprendizado profundo (Deep Learning). A topologia da rede de conhecimento, com densidade de 0,3455 e forte centralidade no cluster \textit{Americas}, sugere um ecossistema de pesquisa coeso, porém regionalmente concentrado, indicando que a validação de algoritmos ainda depende fortemente de contextos neotropicais específicos.

Diferentemente de revisões anteriores que apontavam fragmentação metodológica, os dados atuais indicam uma convergência tecnológica em torno de soluções híbridas e explicáveis. O exame dos 18 estudos de Alta aderência sugere que parte da fronteira do conhecimento tem avançado além da simples classificação de cobertura do solo, aproximando-se de gêmeos digitais com potencial de inferir fitopatologias, dinâmica de carbono e mobilidade humana em condições operacionais reportadas. Contudo, a transposição dessas evidências para decisões de salvaguarda enfrenta o desafio da robustez externa, pois modelos treinados em contextos de alta disponibilidade de dados, carecem de validação sistemática em paisagens de agricultura itinerante na Ásia ou África, limitando sua escalabilidade regulatória.

Em suma, a auditabilidade socioecológica demandada pela governança contemporânea não se limita à acurácia algorítmica, que pode atingir níveis de Alta aderência, mas também envolve a construção de infraestruturas de dados FAIR e a adoção de protocolos de validação que favoreçam a soberania epistêmica das comunidades locais frente à variabilidade climática e territorial. O futuro da pesquisa em SAT tende a convergir para a transição de classificadores estáticos para sistemas adaptativos e auditáveis, com potencial para converter terabytes de dados orbitais em evidências verificáveis de integridade socioecológica.

\section*{Agradecimentos}

Os autores agradecem à Universidade Federal de Sergipe (UFS), à Universidade Estadual de Feira de Santana (UEFS) e ao Instituto Federal de Sergipe (IFS) pelo apoio institucional e infraestrutural que possibilitou esta pesquisa.

\section*{Financiamento}

Esta pesquisa não recebeu financiamento específico de agências de fomento do setor público, comercial ou sem fins lucrativos.

\section*{Compliance with Ethical Standards}

\begin{itemize}
	\item Conflict of Interest: The authors declare no conflict of interest
	\item Ethical Approval: Not applicable
	\item Data Availability: The complete dataset supporting this study, including the bibliographic corpus, analysis scripts, and intermediate results, is publicly available at the Open Science Framework (OSF) under DOI \url{https://doi.org/10.17605/OSF.IO/J7STC}
\end{itemize}

\bibliography{referencias}

\end{document}
