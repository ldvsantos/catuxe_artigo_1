%Version 3.1 December 2024
% See section 11 of the User Manual for version history
%
%%%%%%%%%%%%%%%%%%%%%%%%%%%%%%%%%%%%%%%%%%%%%%%%%%%%%%%%%%%%%%%%%%%%%%
%%                                                                 %%
%% Please do not use \input{...} to include other tex files.       %%
%% Submit your LaTeX manuscript as one .tex document.              %%
%%                                                                 %%
%% All additional figures and files should be attached             %%
%% separately and not embedded in the \TeX\ document itself.       %%
%%                                                                 %%
%%%%%%%%%%%%%%%%%%%%%%%%%%%%%%%%%%%%%%%%%%%%%%%%%%%%%%%%%%%%%%%%%%%%%

%%\documentclass[referee,sn-basic]{sn-jnl}% referee option is meant for double line spacing

%%=======================================================%%
%% to print line numbers in the margin use lineno option %%
%%=======================================================%%

%%\documentclass[lineno,pdflatex,sn-basic]{sn-jnl}% Basic Springer Nature Reference Style/Chemistry Reference Style

%%=========================================================================================%%
%% the documentclass is set to pdflatex as default. You can delete it if not appropriate.  %%
%%=========================================================================================%%

%%\documentclass[sn-basic]{sn-jnl}% Basic Springer Nature Reference Style/Chemistry Reference Style

%%Note: the following reference styles support Namedate and Numbered referencing. By default the style follows the most common style. To switch between the options you can add or remove "Numbered" in the optional parenthesis.
%%The option is available for: sn-basic.bst, sn-chicago.bst%  
 
%%\documentclass[pdflatex,sn-nature]{sn-jnl}% Style for submissions to Nature Portfolio journals
%%\documentclass[pdflatex,sn-basic]{sn-jnl}% Basic Springer Nature Reference Style/Chemistry Reference Style
\documentclass[pdflatex,sn-mathphys-ay]{sn-jnl}% Math and Physical Sciences Author Year Reference Style
%%\documentclass[pdflatex,sn-aps]{sn-jnl}% American Physical Society (APS) Reference Style
%%\documentclass[pdflatex,sn-vancouver-num]{sn-jnl}% Vancouver Numbered Reference Style
%%\documentclass[pdflatex,sn-vancouver-ay]{sn-jnl}% Vancouver Author Year Reference Style
%%\documentclass[pdflatex,sn-apa]{sn-jnl}% APA Reference Style
%%\documentclass[pdflatex,sn-chicago]{sn-jnl}% Chicago-based Humanities Reference Style

%%%% Standard Packages
%%<additional latex packages if required can be included here>

\usepackage{graphicx}%
\usepackage{multirow}%
\usepackage{amsmath,amssymb,amsfonts}%
\usepackage{amsthm}%
\usepackage{mathrsfs}%
\usepackage[T1]{fontenc}%
\usepackage[utf8]{inputenc}%
\usepackage[brazilian]{babel}%
\usepackage{booktabs}%
\usepackage{url}%

% Reduz "poluicao" no log (e evita texto garbado em avisos de hbox no terminal)
\hbadness=10000
\vbadness=10000


\begin{document}

\title{Auditoria Socioecológica Digital em Sistemas Agrícolas Tradicionais, revisão de escopo e estado da arte em aprendizado de máquina}


\author*[1]{\fnm{Catuxe Varjão de Santana} \sur{Oliveira}}
\author[1]{\fnm{XXXXXXXXX} \sur{XXXXXX}}
\author[1]{\fnm{Luiz Diego Vidal} \sur{Santos}}
\author[1]{\fnm{XXXXXXXX} \sur{XXXXXX}}
\author[1]{\fnm{XXXXXXXXXXX} \sur{XXXXXX}}
\author[1]{\fnm{XXXXXXXXXXX} \sur{DiXXXXXX}}

\affil*[1]{\orgname{Universidade Federal de Sergipe - UFS}, \orgaddress{\country{Brasil}}}

\abstract{Sistemas Agrícolas Tradicionais (SAT) podem ser interpretados como sistemas socioecológicos acoplados, nos quais práticas de manejo, diversidade biocultural e condições biofísicas coevoluem em escalas espaço-temporais heterogêneas, o que torna a governança da integridade socioecológica dependente de evidências auditáveis. Este estudo analisa a maturidade técnica do estado da arte em Aprendizado de Máquina para apoiar um gêmeo digital inferencial orientado à auditoria de serviços ecossistêmicos e de indicadores de integridade associados a SAT. Sob diretrizes PRISMA-ScR, processou-se um corpus de 244 estudos (2010--2025) via triagem automatizada de pontuação ponderada (precisão 94,2\%) e validação manual (CCI=0,87). A síntese combinou análises multivariadas e avaliação de governança de dados sob princípios FAIR, além de uma meta-análise de efeitos aleatórios no subconjunto com acurácia e tamanho amostral reportados ($k=16$). A rede de coocorrência apresentou densidade de 0{.}286 e diâmetro igual a 3 no maior componente conexo, com modularidade $Q=0{.}183$, sugerindo especializações temáticas com conectividade interna elevada e acoplamento intercomunidades limitado. A conformidade FAIR foi baixa no agregado, com score total de 18{.}667/100 e assimetria entre dimensões, com Findable 59{.}140\%, Accessible 0{.}645\%, Interoperable 0{.}215\% e Reusable 14{.}667\%, o que restringe reprodutibilidade e contraprova independente. A meta-análise por família algorítmica indicou maior acurácia pooled para Deep Learning, com 92{.}872\% (IC 95\% 89{.}351--95{.}290, $n=5$), enquanto Random Forest e SVM apresentaram 85{.}700\% (IC 95\% 79{.}323--90{.}349, $n=2$) e 85{.}000\% (IC 95\% 79{.}355--89{.}309, $n=2$), respectivamente. A meta-regressão ponderada por ano indicou tendência positiva não conclusiva, com coeficiente 0{.}478 por ano e $p=0{.}158$. Os resultados delimitam que a auditabilidade computacional em SAT depende menos de incrementos marginais de acurácia e mais de protocolos verificáveis de validação, de transparência analítica e de infraestrutura de dados reusável.}

\keywords{Sistemas Agrícolas Tradicionais, Aprendizado de Máquina, Auditoria Socioecológica, Integridade Socioecológica, Rastreabilidade, Serviços Ecossistêmicos}

\maketitle

\section[Introducao]{Introdução}
Sistemas Agrícolas Tradicionais (SAT) têm sido descritos como sistemas socioecológicos acoplados nos quais práticas de manejo, diversidade biocultural e condições biofísicas coevoluem em horizontes temporais longos, sustentando paisagens culturais e serviços ecossistêmicos sob pressões crescentes no Antropoceno \citep{Berkes2003}. Em um cenário marcado por instabilidade climática e erosão de biodiversidade, a persistência desses sistemas tende a depender da capacidade de demonstrar integridade socioecológica em escalas compatíveis com a governança territorial, articulando evidências sobre solo, clima, biota, conhecimento local e práticas de manejo segundo uma lógica de verificabilidade e transparência \citep{Levin1998ComplexAdaptiveSystems}. Embora esse enquadramento forneça um alicerce conceitual para interpretar SAT como sistemas complexos e acoplados, a literatura especializada sobre SAT depende de recortes territoriais e institucionais e permanece heterogênea, de modo que as referências aqui mobilizadas funcionam como suporte teórico geral do argumento, sem pretensão de esgotar o debate específico do campo.

A legitimidade contemporânea de estratégias de salvaguarda e valorização associadas a SAT, quando vinculada a reconhecimento patrimonial, políticas de conservação e instrumentos de rastreabilidade, tende a emergir de arranjos institucionais e comunitários que reduzem assimetrias informacionais, sobretudo quando alegações de sustentabilidade são mobilizadas para acesso a mercados, recursos ou programas públicos \citep{Gale2023}. A credibilidade desses arranjos se sustenta na demonstração empírica de que a continuidade do sistema se ancora em interações socioecológicas específicas, com dependências territoriais nem sempre replicáveis fora do contexto local, e de que variações no manejo e no regime climático não degradam a integridade do conjunto de funções ecossistêmicas e culturais que estrutura o sistema.

Ao serem compreendidos como territórios socioecológicos intrinsecamente acoplados, nos quais solo, clima, biota, cultura e instituições se articulam por interações não lineares e heterogeneidades espaço-temporais, os SAT produzem simultaneamente alimentos, serviços ecossistêmicos e valores bioculturais cuja mensuração depende de escolhas metodológicas, qualidade de dados e coerência operacional entre indicadores \citep{LeFloc2016S}.

A complexidade sistêmica e o caráter difuso dos acoplamentos biofísicos e sociais limitam a detecção de serviços ecossistêmicos por métricas convencionais, o que pode fragilizar a governança de bens comuns e ampliar espaço para práticas de \textit{greenwashing} quando a retórica de sustentabilidade não é acompanhada por evidências verificáveis \citep{Levin1998ComplexAdaptiveSystems,Gale2023}. Dada a insuficiência do instrumental analítico clássico para monitoramento em escala de paisagem \citep{Liao2023}, este estudo propõe um gêmeo digital inferencial orientado a SAT, fundamentado na reconstrução computacional dinâmica de interações entre o genótipo territorial, entendido como condições biofísicas e ecológicas, e o fenótipo socioecológico, entendido como práticas, outputs e assinaturas observáveis do sistema, de modo a converter incerteza ecológica em evidências auditáveis de integridade e conformidade.

A validação operacional desse sistema requer o atendimento a critérios de auditabilidade derivados das lacunas metodológicas identificadas na literatura, e a robustez inferencial torna-se um requisito central, assegurada mediante validação espacialmente independente e estabilidade longitudinal frente à variabilidade climática \citep{Kuhn2013}. A transparência causal, por sua vez, demanda métodos de Inteligência Artificial Explicável (XAI) que permitam discriminar marcadores físico-químicos de correlações espúrias \citep{Rudin2019}. Além disso, a soberania de dados depende da adesão aos princípios FAIR e de trilhas de auditoria imutáveis, que sustentem reprodutibilidade e rastreabilidade da inferência \citep{Wilkinson2021,Kshetri2014}. Nessa arquitetura, o Aprendizado de Máquina opera como mecanismo analítico para tratar a não-linearidade de dados multiescalares, com potencial para apoiar a soberania epistêmica das comunidades produtoras quando acoplado a governança de dados e critérios explícitos de validação \citep{Li2022KGML_ag,Santos2007Epistemologies}.

Em escalas geográficas amplas, o ML pode sustentar a auditabilidade de serviços ecossistêmicos ao estabelecer uma ligação verificável entre conformidade ambiental e prêmio de mercado, mitigando assimetrias informacionais associadas a fraudes e apropriação indevida \citep{Kshetri2014DigitalDivide}.

Ainda assim, a literatura carece de um framework conceitual unificado que integre capacidades inferenciais do ML com requisitos de auditabilidade aplicáveis à governança de SAT, e essa lacuna restringe a tradução de avanços metodológicos em protocolos operacionais, perpetuando fragmentação entre pesquisa acadêmica e implementação territorial.

Nesse enquadramento, o presente estudo analisa criticamente se o aparato metodológico atual de Aprendizado de Máquina reúne condições de robustez inferencial, explicabilidade e governança de dados necessárias para sustentar um gêmeo digital inferencial aplicável a SAT como sistema de auditoria de serviços ecossistêmicos e de integridade socioecológica. Considera-se que a modelagem de acoplamentos não lineares entre variáveis ambientais e ecológicas, representando o genótipo territorial, e assinaturas socioecológicas observáveis do sistema, representando o fenótipo, quando sustentada por validação espaço-temporal rigorosa e mecanismos de segurança e governança de dados, tende a produzir evidências auditáveis compatíveis com decisões de salvaguarda e gestão territorial.

\section[Materiais e Metodos]{Materiais e Métodos}

\subsection[Fluxo de Selecao]{Fluxo de Seleção Bibliográfica (PRISMA)}
O processo de seleção seguiu as diretrizes PRISMA-ScR. A busca inicial recuperou 449 registros combinados das bases Scopus e Web of Science. Após a aplicação de critérios de pontuação automatizada e verificação manual, 244 estudos foram considerados elegíveis para a síntese qualitativa e quantitativa (Figura \ref{fig:prisma}).

\begin{figure}[h]
\centering
\includegraphics[width=0.8\textwidth]{../2-FIGURAS/2-EN/prisma_flowdiagram.png}
\caption{Fluxograma PRISMA-ScR do processo de seleção de estudos.}
\label{fig:prisma}
\end{figure}

A triagem inicial do corpus foi conduzida por meio da aplicação automatizada de uma função de pontuação ponderada sobre os registros recuperados, onde $P_{final}$ representa a pontuação final de seleção, $Q_{met}$ corresponde à qualidade metodológica normalizada (0 a 1), $Q_{tem}$ expressa a relevância temática normalizada (0 a 1) e $Q_{biblio}$ denota o impacto bibliométrico normalizado (0 a 1). Posteriormente, realizou-se verificação manual para consolidação da elegibilidade. A consistência entre avaliadores na etapa manual foi quantificada por coeficiente de correlação intraclasse (CCI=0,87), enquanto o desempenho da triagem automatizada apresentou precisão de 94,2\%.

\subsection[Analises estatisticas]{Análises estatísticas}

O corpus de 244 estudos foi submetido a análises estatísticas descritivas para caracterização bibliométrica. As análises multivariadas, de redes e de agrupamento foram conduzidas a partir de um subconjunto com categorização padronizada, permitindo comparabilidade entre variáveis ao longo do período 2010--2025. Para a síntese quantitativa de desempenho, utilizou-se o subconjunto com acurácia e tamanho amostral explicitamente reportados, o que delimitou a evidência disponível para meta-análise e meta-regressão.

As estratégias analíticas foram articuladas para responder a uma questão de coerência operacional, na qual a maturidade técnica do campo só se torna relevante para SAT quando pode ser conectada a requisitos mínimos de auditabilidade. Nesse sentido, a caracterização temporal e relacional delimita como o campo se organiza e se fragmenta, as sínteses multivariadas e de agrupamento descrevem a heterogeneidade de perfis metodológicos e contextuais, a meta-análise e a meta-regressão testam a estabilidade da evidência quantitativa reportada e sua sensibilidade a regimes de validação, e a avaliação FAIR fornece um diagnóstico de governança que condiciona reprodutibilidade e contraprova independente.

\subsubsection[Analises descritivas e exploratorias do corpus]{Análises descritivas e exploratórias do corpus}

Para a caracterização estrutural do campo científico e mapear as tendências temáticas e os centros de produção de conhecimento, procedeu-se à extração e sistematização de metadados bibliométricos, focando na frequência de palavras-chave e na distribuição geográfica das afiliações institucionais.


A estrutura de interações do domínio foi investigada por meio de Análise de Redes Sociais (ARS) \citep{Schoch2020}. A partir de uma matriz de coocorrências entre valores categóricos extraídos do corpus, construiu-se um grafo não direcionado e ponderado com \texttt{networkx}, no qual nós representam valores categóricos padronizados (por exemplo, famílias algorítmicas, tipos de aplicação, contexto socioecológico e fontes de dado) e arestas representam a frequência de coocorrência entre pares. Para reduzir ruído e tornar comparáveis as estatísticas estruturais reportadas, aplicou-se um limiar mínimo de coocorrência e removeram-se nós isolados.

Para caracterizar a conectividade global, estimaram-se densidade e diâmetro da rede, além de coeficiente de transitividade como proxy de fechamento triádico. Para identificar a importância relativa dos elementos, calcularam-se métricas de centralidade de grau e de intermediação, e, quando pertinente à interpretação, métricas complementares de proximidade e centralidade por autovetor. A estrutura de comunidades foi estimada por maximização gulosa de modularidade, com modularidade reportada como indicador de fragmentação relacional do vocabulário analítico do corpus.

Operacionalmente, o critério de filtragem da rede foi aplicado no nível das arestas, mantendo-se apenas coocorrências com peso igual ou superior a 3 na rede global, o que reduz efeitos espúrios associados a termos raros e torna mais comparáveis as estatísticas estruturais reportadas. Após a filtragem, removeu-se todo nó que permaneceu sem conexões, isto é, nós isolados definidos como vértices com grau zero no grafo resultante. Para examinar interações específicas entre dimensões do corpus, a matriz de coocorrências foi projetada em uma sub-rede por cruzamento entre categorias, preservando apenas pares que conectam classes distintas conforme a taxonomia adotada, com um limiar mínimo de 2 para retenção de arestas nesses recortes, de modo que a sub-rede represente um grafo bipartido no qual as arestas expressam coocorrência entre classes.

A evolução temporal (2010--2025) da produção científica e da adoção de famílias algorítmicas foi analisada mediante testes de tendência, utilizando a correlação de Spearman \citep{Spearman1904} para verificar a monotonicidade das séries temporais.

Para sintetizar associações latentes entre categorias observadas no corpus, conduziu-se Análise de Correspondência Múltipla (MCA) sobre variáveis categóricas, interpretando proximidades no espaço fatorial como similaridade de perfis e não como relações causais. A interpretação do biplot foi conduzida em consonância com a finalidade do estudo, ao tratar a dimensão temporal como marcador de deslocamentos na composição metodológica do campo.

A síntese de perfis metodológicos foi complementada por análise de agrupamento, na qual combinações de atributos do corpus foram resumidas por \textit{k}-means, com seleção de $k$ a partir do critério de Silhouette. Para visualizar padrões relativos à presença média de atributos por grupo, foi gerado um heatmap de perfil de características por cluster.

\subsubsection[Analises inferenciais de validacao dos criterios operacionais]{Análises inferenciais de validação dos critérios operacionais}

Para quantificar a evidência de desempenho disponível e fundamentar requisitos mínimos de validação, conduziu-se meta-análise de efeitos aleatórios \citep{Borenstein2009} no subconjunto de estudos que reportaram acurácia e tamanho amostral, transformando acurácias por logit para estabilização de variâncias \citep{Barendregt2013} usando $\text{logit}(p)=\ln[p/(1-p)]$. Estimou-se acurácia pooled por família algorítmica com estimativa de heterogeneidade por REML, aplicando a retrotransformação para a escala percentual e reportando IC 95\%. Meta-regressão ponderada investigou o efeito do ano de publicação sobre a acurácia, com pesos definidos pelo inverso de uma variância proxy derivada da aproximação binomial em escala percentual.

Por fim, para avaliar a conformidade com princípios de governança de dados abertos, quantificou-se conformidade FAIR mediante score padronizado (0 a 100 pontos) baseado em 12 indicadores binários de \citep{Wilkinson2016}. Cada indicador contribuiu $100/12 \approx 8{.}33$ pontos. Scores foram agregados nas quatro dimensões FAIR por média aritmética. Análise temporal empregou correlação de Spearman \citep{Spearman1904}. Gráficos radar multidimensionais visualizaram perfis FAIR com benchmark da Comissão Europeia (75/100) \citep{EC2018}.

As análises foram implementadas em R \citep{RCoreTeam2024} e Python, com rotinas customizadas para rede de coocorrência, meta-análise, meta-regressão e conformidade FAIR. Empregou-se $\alpha=0{.}05$ como nível de significância quando aplicável.

\subsection[Auditoria Socioecologica Digital como sistema de auditoria inferencial]{Auditoria Socioecológica Digital como sistema de auditoria inferencial}

A partir da análise sistemática do corpus bibliográfico, identificou-se que aplicações de aprendizado de máquina em Sistemas Agrícolas Tradicionais carecem de um framework conceitual que integre capacidades computacionais com requisitos de auditabilidade socioecológica, sobretudo quando o objetivo envolve salvaguarda territorial, integridade de serviços ecossistêmicos e transparência decisória. Para preencher essa lacuna, propõe-se o conceito de Auditoria Socioecológica Digital como sistema de auditoria inferencial derivado empiricamente da revisão.

\section[Resultados e discussao]{Resultados e discussão}\label{sec:resultados}

A análise sistemática do corpus de 244 estudos (2010--2025) sugere a emergência de um campo científico em expansão, cuja maturidade técnica evoluiu de abordagens exploratórias para arquiteturas de inteligência artificial de maior complexidade. A trajetória temporal das publicações, estimada no subconjunto com categorização padronizada para análise temporal, indica crescimento acelerado ao longo do período, partindo de uma produção incipiente em 2010 (5 estudos) e atingindo um volume elevado no biênio 2024--2025 (55 estudos acumulados), sinalizando a consolidação do monitoramento remoto como eixo metodológico recorrente na análise de SAT (Fig. \ref{fig:temporal_combined}a). Essa leitura descritiva é interpretada como um primeiro componente de auditabilidade, pois expõe a base empírica comparável e delimita o ritmo de incorporação de técnicas que podem sustentar inferência reprodutível e verificável.

Na sequência, a discussão integra evidência estrutural e inferencial em uma lógica de encadeamento, na qual conectividade temática, heterogeneidade metodológica, robustez externa e governança de dados são tratadas como dimensões interdependentes para sustentar um gêmeo digital inferencial voltado à auditoria em SAT.

Ao examinar as tendências tecnológicas de forma decomposta (Fig. \ref{fig:temporal_combined}b), qualifica-se esse crescimento quantitativo e sugere-se um ponto de inflexão metodológica situado no triênio 2018--2020. Enquanto a fase inicial foi sustentada por classificadores de aprendizado de máquina convencionais (CML), notadamente \textit{Random Forest} e SVM, cuja recorrência em dados espectrais de média resolução foi compatível com a hegemonia inicial, o período recente (2020--2025) se caracteriza pelo aumento relativo de arquiteturas de \textit{Deep Learning}. 

\begin{figure}[ht]
\centering
\begin{minipage}{0.48\textwidth}
\centering
\includegraphics[width=\linewidth]{../2-FIGURAS/2-EN/temporal_publicacoes.png}
\textbf{(a)}
\end{minipage}\hfill
\begin{minipage}{0.48\textwidth}
\centering
\includegraphics[width=\linewidth]{../2-FIGURAS/2-EN/temporal_algoritmos.png}
\textbf{(b)}
\end{minipage}
\caption{Dinâmica temporal da pesquisa (2010--2025) onde, (a) Evolução exponencial do volume de publicações e (b) Trajetória de substituição tecnológica}
\label{fig:temporal_combined}
\end{figure}


Esta transição é compatível com a complexidade crescente dos dados de observação da Terra e com a pressão por modelos capazes de tratar não linearidades e padrões espaço-temporais em escalas mais finas, contexto em que arquiteturas de \textit{Deep Learning} passam a ser reportadas com maior frequência em tarefas antes dominadas por classificadores marginais. Ainda assim, o deslocamento tecnológico não deve ser interpretado como ganho automático de auditabilidade, pois o uso de modelos mais expressivos aumenta o custo de explicabilidade e tende a ampliar a dependência de desenhos de validação rigorosos e de governança de dados para sustentar generalização fora do cenário de treinamento \citep{Kuhn2013,Rudin2019}.

O resumo dos principais estudos de alta relevância (Tabela \ref{tab:top7}) indica a diversidade de aplicações desde a Amazônia até o Sudeste Asiático. Identificou-se na triagem um núcleo de estudos de Alta aderência (Score $\geq$ 15), que representam o estado da arte na integração entre inteligência computacional e sistemas tradicionais. Destaca-se o trabalho de \cite{Li2025}, que propõe sistemas unificados de AIoT para controle de pragas, e a pesquisa de \cite{Tripathi2025}, que desenvolve arquiteturas híbridas de Deep Learning para diagnóstico fitopatológico.


\begin{table}[h]
\caption{Estudos de Alta aderência sobre em ML aplicado a SAT (2024-2025)}
\label{tab:top7}
\begin{tabular}{@{}llp{8cm}@{}}
\toprule
Ano & Ref & Título e Contribuição \\
\midrule
2025 & \cite{Li2025} & \textit{Unified Pest Prevention...}. Sistema AIoT integrado para controle autônomo e sustentável. \\
2025 & \cite{Tripathi2025} & \textit{Hybrid deep learning system...}. Arquitetura híbrida para classificação de doenças com alta acurácia. \\
2025 & \cite{Ghilardi2025} & \textit{NDVI time series...}. Uso de séries temporais Landsat para monitoramento de degradação em Madagascar. \\
2025 & \cite{Spyrou2025} & \textit{XR-Based Digital Twins...}. Gêmeos digitais imersivos para educação e gestão em agricultura. \\
2025 & \cite{Persson2025} & \textit{Agrarian transitions...}. Análise de sensoriamento remoto sobre transições agrárias no Vietnã/Laos. \\
2024 & \cite{Trehard2024} & \textit{Impact of mobility...}. Correlação entre mobilidade humana e vetores ambientais na Amazônia. \\
2024 & \cite{Li2024} & \textit{Normalized Difference Red-NIR-SWIR...}. Novo índice espectral para mapeamento de agricultura itinerante. \\
\bottomrule
\end{tabular}
\end{table}

\subsection[Panorama das aplicacoes de aprendizado de maquina]{Panorama das aplicações de aprendizado de máquina em Sistemas Agrícolas}

Observa-se na topologia da rede de conhecimento (Figura \ref{fig:network_topology}), caracterizada por uma densidade de 0{.}286 e um diâmetro de 3 (no maior componente conexo), um campo científico densamente conectado, em que nós centrais estruturam o fluxo de informação entre domínios distintos. A visualização global destaca a interconectividade entre famílias algorítmicas, evidências e tipos de aplicação, enquanto a partição em comunidades apresenta modularidade $Q=0{.}183$ (no maior componente conexo), sugerindo especializações temáticas que nem sempre se comunicam. Essa estrutura é relevante para a tese de auditabilidade, porque indica que a evidência computacional tende a se organizar em blocos metodológicos que podem dificultar replicação cruzada e comparação externa quando protocolos e dados não são compartilhados.

\begin{figure}[!htbp]
\centering
\begin{minipage}{0.48\textwidth}
\centering
\includegraphics[width=\linewidth]{_fig_cache/network_completa.png}
{\footnotesize\textbf{(a)}}
\end{minipage}\hfill
\begin{minipage}{0.48\textwidth}
\centering
\includegraphics[width=\linewidth]{_fig_cache/network_communities.png}
{\footnotesize\textbf{(b)}}
\end{minipage}
\caption{Rede de coocorrência de valores categóricos, na qual (a) apresenta a topologia completa e (b) destaca as comunidades.}
\par{\footnotesize\textit{Nota.} As cores indicam comunidades, interpretadas como grupos de valores categóricos que coocorrem com maior frequência entre si do que com o restante da rede. O tamanho dos nós representa o grau, isto é, o número de conexões de cada nó. A espessura das arestas representa o peso, isto é, a frequência de coocorrência entre pares de nós.}\par
\label{fig:network_topology}
\end{figure}

Essa transição é corroborada pela frequência de palavras-chave, na qual \textit{Remote Sensing} (98), \textit{Shifting Cultivation} (49), \textit{Deforestation} (31), \textit{Land Use} (24), \textit{Landsat} (22), \textit{Deep Learning} (20), \textit{Machine Learning} (18), \textit{Land Use Change} (17), \textit{GIS} (15) e \textit{Artificial Intelligence} (13) se destacam, sugerindo concomitantemente a centralidade do sensoriamento remoto como ferramenta e a consolidação do aprendizado profundo como abordagem analítica predominante.

Quando a rede é decomposta em cruzamentos entre categorias, observa-se que as associações não se distribuem de forma uniforme, e a recorrência de determinados pares sugere afinidades pragmáticas entre famílias algorítmicas e tipos de aplicação. O recorte algoritmo\texttimes{}aplicação preserva vínculos suficientes para indicar quais combinações emergem com maior frequência no corpus, ao passo que a síntese por métricas de centralidade permite reconhecer os nós que concentram coocorrências e funcionam como pontos de intermediação entre domínios (Figura \ref{fig:network_crossings}).

\begin{figure}[htbp]
\centering
\begin{minipage}{0.48\textwidth}
\centering
\includegraphics[width=\linewidth]{_fig_cache/network_algoritmo_produto.png}
{\footnotesize\textbf{(a)}}
\end{minipage}\hfill
\begin{minipage}{0.48\textwidth}
\centering
\includegraphics[width=\linewidth]{_fig_cache/network_centrality_metrics.png}
{\footnotesize\textbf{(b)}}
\end{minipage}
\caption{Visões de coocorrência entre categorias no corpus de SAT, nas quais (a) apresenta a sub-rede algoritmo\texttimes{}aplicação e (b) sintetiza métricas de centralidade para os principais nós.}
\label{fig:network_crossings}
\end{figure}

No conjunto dos 18 estudos de "Alta aderência", que concentraram maior impacto e rigor metodológico, delineiam-se as fronteiras dessa transição, e trabalhos seminais recentes, como o de \cite{Li2024}, indicam o potencial de novos índices espectrais frente ao desafio de monitoramento.


\subsection[Analise de Correspondencia Multipla]{Análise de Correspondência Múltiplas (MCA)}

Para compreender as associações latentes entre as categorias metodológicas e geográficas, a Análise de Correspondência Múltipla (MCA) projetou as variáveis no espaço fatorial, com destaque para a dimensão temporal da análise (Figura \ref{fig:mca_temporal}), que elucida a trajetória evolutiva do campo. O biplot revela a estrutura de variância conjunta, onde a proximidade entre pontos indica uma forte associação estatística. Observa-se um deslocamento claro do centroide "2010-2015", associado a técnicas convencionais e monitoramento local, em direção ao centroide "2020-2025", fortemente correlacionado com Big Data, Deep Learning e escalas globais de análise.

Nota-se um agrupamento distinto de técnicas de aprendizado profundo associadas a aplicações de sensoriamento remoto, contrastando com métodos estatísticos clássicos vinculados a dados de levantamento de campo. Ao detalhar as especificidades de cada domínio mediante projeção segregada das categorias, nota-se que certas culturas agrícolas tradicionais, como o cultivo itinerante e sistemas agroflorestais, ocupam nichos específicos no espaço multivariado, exigindo combinações particulares de sensores e algoritmos para sua caracterização adequada. 

\begin{figure}[h]
\centering
\includegraphics[width=1\textwidth]{../2-FIGURAS/2-EN/mca_biplot_temporal_completo.png}
\caption{Biplot temporal, evolução das associações temáticas (2010--2025).}
\label{fig:mca_temporal}
\end{figure}

\subsection[Analise de clusters e perfis]{Análise de clusters e perfis metodológicos}

A estrutura de conectividade já observada na rede de coocorrência, na qual a distribuição de relações entre categorias sugere especializações temáticas no nível relacional, é complementada pela análise de agrupamento, que resume perfis metodológicos por similaridade de atributos. No subconjunto com categorização completa utilizado para o heatmap (2010--2025), a clusterização por \textit{k}-means \citep{MacQueen1967} indicou $k=2$ como número ótimo segundo o critério de Silhouette \citep{Rousseeuw1987}, com Silhouette $=0{.}215$, e produziu uma distribuição assimétrica, com $n=115$ em um grupo e $n=70$ no outro. Essa separação sugere um gradiente de perfis metodológicos e contextuais, mais do que uma taxonomia de nichos de alta resolução, em consonância com a natureza categórica do espaço de atributos que descreve o corpus.
 
Os perfis médios indicam que um dos grupos concentra maior presença de registros associados a contexto geral de SAT e a uso de dados de observação da Terra, enquanto o outro grupo concentra maior presença de registros associados a agricultura itinerante e a tarefas de detecção e monitoramento de mudanças de uso e cobertura da terra, o que é compatível com o deslocamento observado no espaço multivariado do biplot e com a centralidade de séries Landsat e Sentinel-2 no corpus \citep{Li2024,Ghilardi2025}. Embora o agrupamento não descreva desempenho, ele delimita uma estrutura de heterogeneidade operacional que condiciona comparabilidade de métricas e padronização de protocolos.

\begin{figure}[h]
\centering
\includegraphics[width=0.85\textwidth]{../2-FIGURAS/2-EN/cluster_heatmap_profiles_edit.png}
\caption{Heatmap de perfil de características por cluster no subconjunto categorizado do corpus.}
\label{fig:cluster_heatmap_profiles}
\end{figure}


O padrão de assimetria entre os dois grupos é coerente com uma leitura de heterogeneidade metodológica em que perfis operacionais se distribuem ao longo de um gradiente, o que tende a elevar o custo de generalização e a dificultar a definição de protocolos de auditoria aplicáveis a SAT em contextos territoriais diversos quando categorias e metadados não são disponibilizados de forma reusável.

\subsection[Confronto com a Literatura e Implicações para a Governança]{Confronto com a Literatura e Implicações para a Governança}

Ao contrastar estes resultados com revisões anteriores, nota-se uma mudança de paradigma. Enquanto estudos da década passada focavam na viabilidade técnica de sensores isolados \citep{Liakos2018}, a fronteira atual, representada pelos trabalhos de 2024 e 2025, deslocou-se para a integração de sistemas complexos. A emergência de soluções baseadas em AIoT, como proposto por \cite{Li2025}, sugere que a auditoria ambiental não se limitará mais à observação passiva via satélite, mas incorporará redes de sensores in situ para validação em tempo real.

Contudo, a sofisticação tecnológica não elimina fraturas estruturais de governança e de generalização, e a leitura do corpus aponta que o ganho de complexidade algorítmica coexiste com assimetria persistente na abertura e na verificabilidade dos dados. A avaliação de conformidade com princípios FAIR associada à Figura \ref{fig:fair_radar} indica score total de 18{.}667/100, com forte assimetria entre dimensões, com Findable 59{.}140\%, Accessible 0{.}645\%, Interoperable 0{.}215\% e Reusable 14{.}667\%, o que restringe a contraprova independente e amplia incerteza sobre reprodutibilidade. Nesse cenário, a Figura \ref{fig:fair_radar} funciona como proxy de governança e não apenas como diagnóstico descritivo, pois a fragilidade de Accessible e Interoperable reduz a auditabilidade operacional em SAT ao impedir que a evidência computacional seja reusada em ciclos independentes de validação.


\begin{figure}[h]
\centering
\includegraphics[width=1\textwidth]{../2-FIGURAS/2-EN/fair_radar_2.png}
\caption{Conformidade com princípios FAIR de governança de dados, (a) radar FAIR e (b) conformidade por indicador.}
\label{fig:fair_radar}
\end{figure}


Sem a disponibilização aberta de códigos, dados de treinamento e metadados, a reprodutibilidade alegada em muitos estudos torna-se inverificável, criando o que \cite{Shakeripour20241257} identificam como um falso otimismo na maturidade da IA aplicada a sistemas agroecológicos. A dependência de caixas pretas algorítmicas se torna ainda mais sensível quando a síntese quantitativa evidencia simultaneamente alto desempenho médio e forte heterogeneidade. 

No modelo de efeitos aleatórios, a síntese quantitativa foi restrita ao subconjunto com acurácia e tamanho amostral reportados, o que delimitou $k=16$ estudos com evidência numérica comparável para meta-análise e meta-regressão. A estratificação por família algorítmica indica acurácia pooled mais elevada em Deep Learning, com 92{.}872\% (IC 95\% 89{.}351--95{.}290, $n=5$), seguida por um grupo residual agregado como Other, com 87{.}440\% (IC 95\% 82{.}812--90{.}959, $n=5$), enquanto Random Forest e SVM apresentaram 85{.}700\% (IC 95\% 79{.}323--90{.}349, $n=2$) e 85{.}000\% (IC 95\% 79{.}355--89{.}309, $n=2$), respectivamente (Fig. \ref{fig:meta_algoritmo}a). Esse gradiente de desempenho deve ser interpretado em conjunto com o tamanho amostral por algoritmo, pois intervalos de confiança amplos emergem quando $n$ é pequeno e reduzem a força inferencial de comparações diretas.


\begin{figure}[htbp]
\centering
\begin{minipage}[t]{0.49\textwidth}
\centering
\includegraphics[width=\textwidth]{../2-FIGURAS/2-EN/meta_analise_algoritmos.png}
\par\small (a)
\end{minipage}
\hfill
\begin{minipage}[t]{0.49\textwidth}
\centering
\includegraphics[width=\textwidth]{../2-FIGURAS/2-EN/meta_regressao_ano.png}
\par\small (b)
\end{minipage}
\caption{(a) Síntese meta-analítica das acurácias por algoritmo; (b) meta-regressão que examina a tendência temporal do desempenho reportado ao longo do período de publicação.}
\label{fig:meta_algoritmo}
\end{figure}

Em paralelo, a meta-regressão ponderada não indicou tendência temporal conclusiva do desempenho, pois o coeficiente estimado por ano foi 0{.}478 e $p=0{.}158$ (Fig. \ref{fig:meta_algoritmo}b). Essa evidência é consistente com um cenário em que ganhos reportados de acurácia não evoluem de forma monotônica ao longo do tempo, e sugere que variações na disponibilidade de dados, no desenho de validação e na definição operacional dos alvos podem dominar a variância observada em métricas agregadas.

A validade ecológica dos modelos também enfrenta o desafio da falácia da escala. A eficácia demonstrada por \cite{Li2024} na detecção de clareiras de agricultura itinerante com novos índices espectrais (Red-NIR-SWIR) é promissora, mas sua aplicação em biomas distintos, a exemplo das florestas de Miombo analisadas por \cite{Andrews2024} ou das estepes da Mongólia estudadas por \cite{Prudnikova2024496}, carece de testes de robustez. A degradação de desempenho observada em testes externos sugere que modelos treinados em nichos ecológicos específicos podem apresentar queda sistemática de desempenho ao serem aplicados a paisagens biogeograficamente distintas, sobretudo quando o deslocamento espacial altera o regime espectral, a fenologia e o contexto de manejo.

Isso é compatível com a tese de \cite{Ghilardi2025} de que a auditoria de habitats críticos tende a exigir calibração local intensiva, o que relativiza a ideia de "algoritmos universais" para monitoramento de biodiversidade.

\section[Conclusoes]{Conclusões}

Esta revisão de escopo sintetiza evidências de que a pesquisa em aprendizado de máquina aplicada a Sistemas Agrícolas Tradicionais tem se expandido e sofisticado, com deslocamento recente para arquiteturas de maior complexidade e para integração de múltiplas fontes de dados. A literatura também sugere que esse avanço técnico não se traduz automaticamente em maturidade operacional para auditoria socioecológica, porque persistem barreiras de transferibilidade entre contextos territoriais, assimetrias de governança e dependência de escolhas metodológicas que afetam a robustez do desempenho fora do cenário de treinamento.

Do ponto de vista metodológico, o corpus analisado converge para um núcleo de abordagens quantitativas que combina caracterização bibliométrica, análise de redes de coocorrência para descrever relações entre termos, estatística multivariada para sintetizar associações latentes, agrupamento para resumir perfis metodológicos, meta-análise e meta-regressão para integrar métricas de desempenho, e avaliação de governança de dados sob princípios FAIR. Essa combinação de métodos aponta que a estrutura do campo, a heterogeneidade de desenhos e a disponibilidade de evidência computacional são dimensões interdependentes, sobretudo quando a finalidade é sustentar inferência auditável.

Quanto aos principais materiais e evidências mobilizados nos estudos sobre SAT, predomina o uso de dados de observação da Terra, com imagens e séries temporais de sensoriamento remoto e índices espectrais, frequentemente articulados a atributos de campo e a sensores espectrais de proximidade em desenhos voltados à classificação, detecção e monitoramento. As aplicações se distribuem entre tarefas associadas a sistemas agroflorestais e agricultura itinerante, mapeamento de cobertura e uso do solo em cultivos anuais, e diagnóstico em culturas perenes, com recorrência de famílias algorítmicas como Random Forest, SVM e arquiteturas de aprendizado profundo.

O conjunto de resultados sugere que métricas de desempenho elevadas podem coexistir com heterogeneidade substantiva entre estudos e com evidência quantitativa limitada a um subconjunto pequeno de trabalhos com reporte adequado, o que restringe a força inferencial de conclusões baseadas apenas em agregação de acurácias. Em paralelo, a conformidade com princípios de governança de dados permanece limitada em parte relevante do corpus, restringindo reprodutibilidade, contraprova independente e rastreabilidade, elementos centrais quando o uso pretendido envolve decisões regulatórias e salvaguarda territorial.

A contribuição do estudo consiste em delimitar que a auditabilidade computacional em SAT tende a depender menos de incrementos marginais de acurácia e mais de protocolos de validação coerentes com o território, de transparência analítica e de infraestrutura de dados que permita reuso e verificação. A agenda de pesquisa e implementação se beneficia da consolidação de benchmarks de integridade socioecológica e de desenhos de avaliação que incorporem deslocamentos espaciais e temporais, de modo que o gêmeo digital inferencial seja interpretável e defensável em cenários socioecológicos diversos.

\section*{Agradecimentos}

Os autores agradecem à Universidade Federal de Sergipe (UFS), à Universidade Estadual de Feira de Santana (UEFS) e ao Instituto Federal de Sergipe (IFS) pelo apoio institucional e infraestrutural que possibilitou esta pesquisa.

\section*{Conflitos de interesse}

Os autores declaram não haver conflitos de interesse.

\section*{Declaração de disponibilidade de dados}

O conjunto de dados completo que apoia os resultados deste estudo, incluindo o corpus bibliográfico, os scripts de análise e os resultados intermediários, está disponível publicamente no repositório Open Science Framework (OSF) sob DOI \url{https://doi.org/10.17605/OSF.IO/2EKYQ}.

\bibliography{referencias}

\end{document}
