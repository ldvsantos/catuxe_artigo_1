%Version 3.1 December 2024
% See section 11 of the User Manual for version history
%
%%%%%%%%%%%%%%%%%%%%%%%%%%%%%%%%%%%%%%%%%%%%%%%%%%%%%%%%%%%%%%%%%%%%%%
%%                                                                 %%
%% Please do not use \input{...} to include other tex files.       %%
%% Submit your LaTeX manuscript as one .tex document.              %%
%%                                                                 %%
%% All additional figures and files should be attached             %%
%% separately and not embedded in the \TeX\ document itself.       %%
%%                                                                 %%
%%%%%%%%%%%%%%%%%%%%%%%%%%%%%%%%%%%%%%%%%%%%%%%%%%%%%%%%%%%%%%%%%%%%%

%%\documentclass[referee,sn-basic]{sn-jnl}% referee option is meant for double line spacing

%%=======================================================%%
%% to print line numbers in the margin use lineno option %%
%%=======================================================%%

%%\documentclass[lineno,pdflatex,sn-basic]{sn-jnl}% Basic Springer Nature Reference Style/Chemistry Reference Style

%%=========================================================================================%%
%% the documentclass is set to pdflatex as default. You can delete it if not appropriate.  %%
%%=========================================================================================%%

%%\documentclass[sn-basic]{sn-jnl}% Basic Springer Nature Reference Style/Chemistry Reference Style

%%Note: the following reference styles support Namedate and Numbered referencing. By default the style follows the most common style. To switch between the options you can add or remove Numbered in the optional parenthesis. 
%%The option is available for: sn-basic.bst, sn-chicago.bst%  
 
%%\documentclass[pdflatex,sn-nature]{sn-jnl}% Style for submissions to Nature Portfolio journals
%%\documentclass[pdflatex,sn-basic]{sn-jnl}% Basic Springer Nature Reference Style/Chemistry Reference Style
%%\documentclass[pdflatex,sn-mathphys-ay]{sn-jnl}% Math and Physical Sciences Author Year Reference Style
%%\documentclass[pdflatex,sn-aps]{sn-jnl}% American Physical Society (APS) Reference Style
%%\documentclass[pdflatex,sn-vancouver-num]{sn-jnl}% Vancouver Numbered Reference Style
%%\documentclass[pdflatex,sn-vancouver-ay]{sn-jnl}% Vancouver Author Year Reference Style
% Prevents pdfTeX warnings about duplicate hyperlink destinations (figure.1, table.1, etc.)
% when hyperref is loaded by the sn-jnl class.
\PassOptionsToPackage{hypertexnames=false}{hyperref}
\documentclass[lineno,pdflatex,sn-apa]{sn-jnl}% APA Reference Style
%%\documentclass[pdflatex,sn-chicago]{sn-jnl}% Chicago-based Humanities Reference Style

%%%% Standard Packages
\usepackage{graphicx}%
\usepackage{multirow}%
\usepackage{amsmath,amssymb,amsfonts}%
\usepackage{amsthm}%
\usepackage{mathrsfs}%
\usepackage[T1]{fontenc}%
\usepackage[utf8]{inputenc}%
\usepackage[american]{babel}%
\usepackage{booktabs}%
\usepackage{url}%
\usepackage{verbatim}
\usepackage{float}%

\begin{document}

\title{\textbf{Machine Learning Readiness to Support Agroecological Transitions in Traditional Agricultural Systems, a Meta-Analysis}}

\author[1]{\fnm{Catuxe Varjão de Santana} \sur{Oliveira}}
\author*[2]{\fnm{Luiz Diego Vidal} \sur{Santos}}
\author[1]{\fnm{Paulo Roberto} \sur{Gagliardi}}
\author[3]{\fnm{Renisson Neponuceno de} \sur{Araújo Filho}}
\author[2]{\fnm{Tânia Cristina} \sur{Azevedo}}
\author[1]{\fnm{Gustavo da Silva} \sur{Quirino}}
\author[1]{\fnm{Francisco Sandro Rodrigues} \sur{Holanda}}

\email{ldvsantos@uefs.br}

\affil[1]{\orgname{Universidade Federal de Sergipe (UFS)}, \orgaddress{\country{Brazil}}}
\affil*[2]{\orgname{Universidade Estadual de Feira de Santana (UEFS)}, \orgaddress{\country{Brazil}}}
\affil[3]{\orgname{Universidade Federal Rural de Pernambuco (UFRPE)}, \orgaddress{\country{Brazil}}}

\abstract{Traditional Agricultural Systems are coupled social-ecological systems whose integrity requires evidence-based monitoring, yet the operational readiness of machine learning for this purpose remains uncertain. This meta-analysis assesses the extent to which deficiencies in data governance and spatial validation constrain the regulatory applicability of machine learning models in these systems despite their reported high accuracy. We examined 244 studies (2010--2025) retrieved from Scopus and Web of Science following standardized scoping review guidelines, using automated relevance scoring (precision 94.2\%) with manual validation (intraclass correlation coefficient = 0.87), random-effects modeling, network analysis (modularity $Q = 0.62$), and compliance assessment against the Findable, Accessible, Interoperable, and Reusable (FAIR) data principles. Pooled accuracy reached 90.7\% (95\% confidence interval 89.8 to 91.5\%) with substantial heterogeneity ($I^{2} = 58\%$), while deep learning adoption increased from 4.8\% (2015--2019) to 21.1\% (2020--2025). However, FAIR compliance remained critically low (mean score 18.7/100), with only 1.1\% of studies sharing code or data, and publication-bias analysis (Egger's test, $p = 0.009$) suggests performance overestimation by approximately 1.5 percentage points. Critical gaps in spatial validation protocols, algorithmic explainability, and data-governance infrastructure limit operational readiness for regulatory applications in Traditional Agricultural Systems, with direct implications for monitoring heritage sites under the Globally Important Agricultural Heritage Systems program and Sustainable Development Goal targets 2.4 and 15.4. These findings underscore the need for inter-regional validation and transparent algorithmic governance to support evidence-based agrobiodiversity management.}

\keywords{deep learning, machine learning, meta-analysis, data governance, FAIR}

\maketitle

\section[Introduction]{Introduction}

Traditional Agricultural Systems are social-ecological systems in which management practices, biocultural diversity, and biophysical conditions co-evolve over long temporal horizons, sustaining cultural landscapes and ecosystem services under increasing Anthropocene pressures \citep{Berkes2003}. At the global scale, these systems are recognized by institutional frameworks such as the Globally Important Agricultural Heritage Systems (GIAHS) program of the Food and Agriculture Organization of the United Nations, which has designated over 80 sites in 24 countries \citep{Koohafkan2011}, and by the Intergovernmental Science-Policy Platform on Biodiversity and Ecosystem Services (IPBES), which identifies indigenous and local knowledge as indispensable for biodiversity assessment \citep{IPBES2019}.

These systems directly underpin multiple Sustainable Development Goals (SDGs), notably Goal~2 (Zero Hunger), target~2.4 on sustainable food production systems, Goal~13 (Climate Action), and Goal~15 (Life on Land) \citep{UN2015}. Despite this policy relevance, evidence-based landscape monitoring of traditional system integrity remains fragmented across regions and disciplines. Territorial heterogeneity, systemic complexity, and diffuse ecosystem service provision create measurement and monitoring challenges that can be partially mitigated by machine learning integrated with remote sensing data \citep{Weiss2020,Osco2021}.

However, the operational readiness of these technologies to support agroecological transitions in traditional contexts remains uncertain, particularly given data governance and spatial validation gaps that may compromise regulatory applicability and evidence-based accountability. This complexity appears across diverse Traditional Agricultural System typologies, from irrigated terraces and stratified agroforestry systems to agroecological homegardens (Figure \ref{fig:photo_intro}) and communal rangeland systems (\textit{fundo de pasto}) \citep{Koohafkan2011,Altieri2004}, where overlapping tree canopy strata, food crops, livestock, and remnant native vegetation configure land-use mosaics whose social-ecological coupling challenges conventional monitoring.

\begin{figure}[ht]
\centering
\includegraphics[width=0.85\textwidth]{sat.jpg}
\caption{Agroecological homegardens in a Traditional Agricultural System (Photo credit by Catuxe Oliveira).}
\label{fig:photo_intro}
\end{figure}

As intrinsically coupled social-ecological territories in which soil, climate, biota, culture, and institutions interact through nonlinear dynamics and spatiotemporal heterogeneity, these systems simultaneously produce food, ecosystem services, and biocultural values whose measurement depends on methodological choices, data quality, and coherence among indicators \citep{LeFloc2016S}.

The systemic complexity and diffuse character of these couplings entail nonlinearity, context dependence, and multiscalarity, reducing ecosystem service identifiability when conventional metrics and isolated proxies are used. Different social-ecological states may produce similar indicator signals, and the same indicator may respond differently under distinct management regimes and biophysical conditions, weakening inference and accountability in common-pool resource governance. Within this verification gap, sustainability narratives may gain traction without auditable empirical backing, expanding the space for \textit{greenwashing} \citep{Levin1998ComplexAdaptiveSystems,Gale2023}.

Given the limitations of the classical landscape-scale analytical toolkit \citep{Liao2023,Weiss2020}, a framework integrating spatiotemporal observations and validation criteria compatible with territorial heterogeneity is warranted \citep{Belgiu2016}.

Machine learning emerges as a promising tool to implement evidence-based monitoring and decision-support systems by integrating remote sensing data, edaphoclimatic variables, and social-ecological indicators \citep{Mountrakis2011,Spyrou2025}. System robustness benefits from spatially independent validation, longitudinal stability under climatic variability \citep{Kuhn2013}, and explainable artificial intelligence methods that support interpretation of relevant biophysical markers \citep{Rudin2019}. Adherence to FAIR (Findable, Accessible, Interoperable, Reusable) principles underpins analysis reproducibility and traceability \citep{Wilkinson2016}.

Against this background, this scoping review assesses the technical maturity of machine learning applied to Traditional Agricultural Systems, examining reported predictive performance across algorithmic families and application domains (land cover, soil properties, crop yield), the extent of spatial validation strategies that sustain inter-regional generalizability, compliance with FAIR principles in data and code sharing, and the thematic clusters and technological trajectories structuring field evolution between 2010 and 2025.

\section[Materials and Methods]{Materials and Methods}

\subsection[Protocol and Registration]{Protocol and Registration}

The protocol and reproducibility artifacts, including search strings, the bibliographic corpus exported in BibTeX format, screening scripts, and analytical routines, were publicly deposited on the Open Science Framework (OSF) under DOI \url{https://doi.org/10.17605/OSF.IO/J7STC}. The selection process followed PRISMA-ScR guidelines \citep{Tricco2018b}.

\subsection[Information Sources and Search Strategy]{Information Sources and Search Strategy}

Information sources comprised the Scopus and Web of Science databases. The most recent search and export execution occurred on January 24, 2026, as recorded in the BibTeX export metadata files. Corpus coverage was restricted to 2010--2025. No supplementary manual searches were conducted on websites, in institutional repositories, or in reference lists, and no authors were contacted for additional evidence.

The electronic strategy retrieved evidence in which explicit characterization of Traditional Agricultural Systems was jointly associated with machine learning and remote sensing, searching titles, abstracts, and keywords. The Scopus query is reproduced in full below, preserving the temporal limits used for reproducibility, with line breaks for readability.

{\small
\begin{verbatim}
TITLE-ABS-KEY(( "traditional agricultural system*" OR
"traditional farming system*" OR "traditional agriculture"
OR "traditional agroecosystem*" OR "indigenous farming" OR
"indigenous agriculture" OR "shifting cultivation" OR
"slash-and-burn" OR "ancestral farming" OR
"ancient agriculture") AND ( "machine learning" OR
"artificial intelligence" OR "deep learning" OR
"random forest" OR "neural network*" OR
"support vector machine*" OR "SVM" OR "decision tree*" OR
"gradient boosting" OR "CNN" OR
"long short-term memory" OR "LSTM" OR "yolo" OR
"remote sensing")) AND PUBYEAR > 2009 AND PUBYEAR < 2026
\end{verbatim}
}

\subsection[Eligibility Criteria]{Eligibility Criteria}

Eligible studies were journal articles indexed in the consulted databases, retrieved by the search strings within the temporal interval, and containing sufficient metadata for charting, including at least a title and abstract. No explicit language restriction was applied at the search stage, but the strategy used only English-language terms, the dominant indexing language in Scopus and Web of Science, ensuring comparability with international systematic reviews and reducing coverage ambiguity associated with multilingual descriptors. Restricting the evidence universe to these two databases excludes gray literature and non-indexed repositories, a decision adopted to maximize traceability, metadata standardization, and analytical reproducibility.

\subsection[Selection Process and Data Extraction]{Selection Process and Data Extraction}

The initial search retrieved 449 combined records from the Scopus and Web of Science databases. After applying automated scoring criteria and manual verification, 244 studies were deemed eligible for qualitative and quantitative synthesis (Figure \ref{fig:prisma}).

\begin{figure}[ht]
\centering
\includegraphics[width=0.8\textwidth]{prisma_flowdiagram.png}
\caption{PRISMA 2020 flow diagram of the study selection process.}
\label{fig:prisma}
\end{figure}

Initial corpus screening applied an automated weighted scoring function to retrieved records, where $P_{final}$ represents the final selection score, $Q_{met}$ corresponds to normalized methodological quality (0 to 1), $Q_{tem}$ expresses normalized thematic relevance (0 to 1), and $Q_{biblio}$ denotes normalized bibliometric impact (0 to 1). Manual verification then consolidated eligibility. Inter-rater consistency at the manual stage was quantified by the intraclass correlation coefficient (ICC = 0.87), and automated screening performance showed 94.2\% precision.

Although $Q_{met}$ is an operational screening criterion for minimum methodological consistency and reporting completeness, it is not a formal critical appraisal of included sources and was not used to weight estimates, adjust variances, or grade risk of bias in the synthesis.

\subsection[Synthesis Methods]{Synthesis Methods}

\subsubsection[Bibliometric Analysis]{Bibliometric Analysis}

The domain interaction structure was investigated through Social Network Analysis (SNA). An undirected weighted graph was constructed from a categorical corpus dataset, where nodes represent entities (algorithms, instruments, products) and edges indicate co-occurrence across studies, with implementation in Python.

Centrality metrics (degree and betweenness) were calculated to identify the relative importance of network elements. Thematic community detection used modularity maximization on the weighted graph, enabling identification of technological modules and functional specialization patterns. The temporal evolution (2010--2025) of scientific output and algorithmic family adoption was analyzed through time series and descriptive statistics, characterizing trends and inflection points.

\subsubsection[Quantitative Synthesis (Meta-Analysis)]{Quantitative Synthesis (Meta-Analysis)}

For the quantitative synthesis, a subset of 148 studies reporting performance metrics was selected, of which 129 presented recoverable accuracy for meta-analysis. Accuracies were extracted from the reported text (title/abstract/keywords) and standardized as proportions, with logit transformation for variance stabilization and numerical truncation at the boundaries ($\varepsilon=10^{-4}$) when necessary.

Per-study variance was approximated using a binomial model ($p(1-p)/n$) when $n$ was available; when sample size was not reported in a recoverable manner, $n=100$ was adopted as a conservative approximation to enable weighting and visualization. Pooled estimates were obtained using a random-effects model with heterogeneity estimated by restricted maximum likelihood (REML), as follows
\begin{equation}
\hat{\theta} = \frac{\sum_{i=1}^{k} w_i \theta_i}{\sum_{i=1}^{k} w_i}
\end{equation}
where $w_i = 1/(\sigma_i^2 + \tau^2)$ and $\tau^2$ represents the between-study variance \citep{Viechtbauer2010}. Uncertainty was reported as a 95\% confidence interval (CI). Asymmetries consistent with publication bias were examined using Egger's regression, rank correlation, and the \textit{trim-and-fill} procedure, with interpretation as reporting sensitivity diagnostics.

The temporal component of performance was examined through inverse-variance weighted regression with publication year as a covariate, prioritizing interpretation as a reporting trend (rather than a causal effect).

\subsubsection[Multivariate Analysis (MCA and Clustering)]{Multivariate Analysis (MCA and Clustering)}

The associations among methodological, geographic, and temporal categories in the charted corpus were explored through Multiple Correspondence Analysis (MCA) on a categorical table, using the dimensions \textit{Algorithm}, \textit{Evidence}, \textit{Context}, \textit{Application}, and \textit{Region}, with projection onto two factorial dimensions for reading the latent structure of co-associations.

In MCA, categorical variables were encoded as an indicator matrix, and proximities were interpreted under the chi-squared metric inherent to Correspondence Analysis, with normalization preserving comparability among modalities and enabling interpretation of factorial inertias as reduced-space dispersion decomposition \citep{Greenacre2017,Abdi2014}. Computation was performed in Python using the \textit{prince} library (v0.16.2), fitting the complete indicator matrix with reproducible decomposition under the \textit{sklearn} engine (\textit{n\_iter}=10, \textit{random\_state}=42). Factorial projection was reported on the first two dimensions for interpretability and because they concentrated the largest inertia shares, with eigenvalues of 0.5685 and 0.3545 for Dim1 and Dim2, corresponding to 9.17\% and 5.72\% of inertia, with 14.89\% cumulative. The two-dimensional projection captured 14.9\% of total inertia, expected for categorical data with 5 variables \citep{Greenacre2017}, and maintained an interpretable association structure validated by hierarchical clustering with consistent groupings.

Functional groupings in the categorical corpus were identified by $k$-means clustering on a one-hot matrix derived from \textit{Algorithm}, \textit{Evidence}, \textit{Context}, \textit{Application}, and \textit{Region}. The optimal number of clusters ($k$) was determined by maximizing mean silhouette over $k \in [2, 5]$, with $k=2$ (silhouette=0.215) selected to balance interpretability and internal cohesion.

Each cluster profile was summarized by mean occurrence of the 18 most frequent features (selected by global corpus frequency). The hierarchy was estimated by Ward's linkage on a Euclidean distance matrix among features and visualized as a heatmap with a lateral dendrogram, with color intensity from 0 (absent) to 1 (present in 100\% of studies in the cluster) on a blue-purple pastel scale consistent with the manuscript palette.

\subsubsection[FAIR Compliance Assessment]{FAIR Compliance Assessment}

In the data governance axis, FAIR compliance was quantified by a standardized score (0 to 100 points) based on 12 binary indicators \citep{Wilkinson2016}. Indicators were coded from textual signals in metadata and reporting (e.g., DOI, repository presence, code availability, explicit license, documentation), with uniform contribution ($100/12 \approx 8.33$ points per indicator) and aggregation across the four FAIR dimensions by arithmetic mean.

\section[Results]{Results}\label{sec:resultados}

Systematic analysis of the 244-study corpus (2010--2025) indicates a field whose technical maturity is evolving from exploratory approaches toward high-complexity artificial intelligence architectures (Fig. \ref{fig:temporal_combined}a). Publication trajectories suggest exponential growth, rising from incipient output in 2010 (10 studies) to robust output in the 2024--2025 biennium (65 cumulative studies), signaling consolidation of remote monitoring as a central paradigm in agroecosystem management \citep{Weiss2020,Osco2021}.

Disaggregated technology trends (Fig. \ref{fig:temporal_combined}b) identify a methodological inflection in the 2018--2020 triennium. The initial phase was sustained by conventional classifiers, notably \textit{Random Forest} and SVM \citep{Belgiu2016,Mountrakis2011}, whose efficacy with medium-resolution spectral data contributed to early predominance, whereas the recent period (2020--2025) is marked by increasing adoption of \textit{deep learning} architectures \citep{Osco2021,Weiss2020}.

\begin{figure}[ht]
\centering
\begin{minipage}{0.48\textwidth}
\centering
\includegraphics[width=\linewidth]{temporal_publicacoes.png}
\textbf{(a)}
\end{minipage}\hfill
\begin{minipage}{0.48\textwidth}
\centering
\includegraphics[width=\linewidth]{temporal_algoritmos.png}
\textbf{(b)}
\end{minipage}
\caption{Temporal dynamics of the research field (2010--2025). (a) Exponential growth in publication volume. (b) Technological substitution trajectory.}
\label{fig:temporal_combined}
\end{figure}

A core group of high-adherence studies (Score $\geq$ 12) was identified during screening, comprising 18 studies distributed between 2010 and 2025 and predominantly indexed in Scopus (17/18), consistent with the recent frontier of integration between computational intelligence and traditional systems. Notably, \cite{Li2025} proposes unified AIoT systems for autonomous pest control, and \cite{Tripathi2025} develops hybrid deep learning architectures for phytopathological diagnosis.

The top high-relevance studies (Table \ref{tab:top7}) illustrate geographic and thematic diversity across tropical, subtropical, and dryland contexts on multiple continents.

\begin{table}[h]
\caption{High-adherence studies in machine learning applied to traditional agricultural systems (2024--2025)}
\label{tab:top7}
\centering
\small
\setlength{\tabcolsep}{4pt}
\renewcommand{\arraystretch}{1.1}
\begin{tabular}{@{}p{1.2cm}p{2.3cm}p{\dimexpr\linewidth-3.5cm\relax}@{}}
\toprule
Year & Ref & Contribution \\
\midrule
2025 & \cite{Li2025} & Integrated AIoT system for autonomous and sustainable pest control. \\
2025 & \cite{Tripathi2025} & Hybrid architecture for high-accuracy disease classification. \\
2025 & \cite{Ghilardi2025} & Landsat time series for degradation monitoring in Madagascar. \\
2025 & \cite{Spyrou2025} & Immersive digital twins for agricultural education and management. \\
2025 & \cite{Persson2025} & Remote sensing analysis of agrarian transitions in Vietnam/Laos. \\
2024 & \cite{Trehard2024} & Correlation between human mobility and environmental vectors in the Amazon. \\
2024 & \cite{Li2024} & Spectral index for shifting cultivation mapping. \\
\bottomrule
\end{tabular}
\end{table}

\subsection[Overview of machine learning applications]{Overview of machine learning applications in agricultural systems}

The knowledge-network topology (Figure \ref{fig:network_completa}), characterized by a density of 0.345, reveals a densely connected field in which central nodes structure information flow between domains. The global visualization highlights interconnectivity between computational methods and agricultural study objects.

\begin{figure}[H]
\centering
\includegraphics[width=0.65\textwidth]{network_completa.png}
\caption{Knowledge network topology and term co-occurrence.}
\label{fig:network_completa}
\end{figure}

A deeper structural analysis based on community detection (Figure \ref{fig:network_communities}) suggests well-defined thematic clusters. Centrality metrics point to co-prominence of the \textit{Americas}, \textit{Asia}, and \textit{Global} clusters, reflecting geographically distributed yet unequal application of geotechnologies in traditional agricultural landscapes. These regional clusters are strongly coupled with nodes such as \texttt{NeuralNetwork} and \texttt{DeepLearning}, evidencing convergence toward advanced computational approaches despite regional disparities in data availability and validation infrastructure.

This configuration suggests a methodological transition in which the first decade (2010--2020) focused on spectral sensor validation (NIR/FTIR), while the recent period (2021--2025) shows greater participation of deep learning architectures applied to multi-sensor data fusion \citep{Liakos2018,Weiss2020,Osco2021}.

\begin{figure}[H]
\centering
\IfFileExists{louvain_modules_detailed.png}{%
\includegraphics[width=0.90\textwidth]{louvain_modules_detailed.png}%
}{%
\fbox{Figure not found, louvain\_modules\_detailed.png}%
}
\caption{Technological modules identified in the co-occurrence network.}
\label{fig:network_communities}
\end{figure}

The modular structure of the network, estimated by the Louvain algorithm with modularity optimization ($Q = 0.183$), reveals two technological modules (Figure \ref{fig:network_communities}). The topology exhibits small-world properties (clustering coefficient C=0.68 vs. C$_{random}$=0.35; mean path length L=2.4 vs. L$_{random}$=2.1) \citep{Watts1998}, indicating well-defined thematic communities with short inter-community distances. The first module, designated \textit{Deep Learning Module}, concentrates deep neural architectures (CNN, LSTM, Transformer) coupled with high-resolution sensors (Sentinel-2, PlanetScope) and high temporal-frequency monitoring applications.

Within this cluster, strong co-occurrence between \textit{DeepLearning}, \textit{Application=Monitoring}, and \textit{Region=Asia} supports the hypothesis of regional specialization in advanced computational techniques. The second module, characterized as \textit{Technology Module}, aggregates classical ensemble methods (Random Forest, SVM) associated with medium-resolution multispectral products (Landsat, MODIS) and land use and land cover mapping as the dominant application domain in shifting cultivation contexts. Modular segregation indicates that coexistence of methodological paradigms does not imply integration, but functional partitioning in which each module maintains internal coherence and operates within specific niches \citep{Weiss2020,Osco2021}.

This transition is corroborated by keyword frequency, where \textit{Remote Sensing = 98} and \textit{Shifting Cultivation = 49} emerge as dominant terms, indicating the preferred tool and the predominant analytical approach, respectively.

\subsection[Multiple Correspondence Analysis]{Multiple Correspondence Analysis (MCA)}

To elucidate latent associations among methodological and geographic categories, Multiple Correspondence Analysis (MCA) \citep{Greenacre2017,Abdi2014} projected variables into factorial space, emphasizing the temporal dimension (Figure \ref{fig:mca_temporal}) that captures field evolution. The biplot reveals joint variance structure, where proximity between points indicates strong statistical association \citep{Greenacre2017}. A clear displacement is observed from the 2010--2015 centroid, associated with conventional techniques and local monitoring, toward the 2020--2025 centroid, strongly correlated with Big Data, deep learning, and global-scale analyses.

\begin{figure}[H]
\centering
\includegraphics[width=0.85\textwidth]{mca_biplot_temporal_completo.png}
\caption{Temporal biplot showing the evolution of thematic associations (2010--2025).}
\label{fig:mca_temporal}
\end{figure}

Within the high-adherence subset (n=18), a recent concentration was observed, with 10 studies published between 2020 and 2025, indicating that field intensification is not limited to volumetric growth but also includes higher information density in monitoring and inference approaches (Figure \ref{fig:mca_temporal}). Under this regime, the technical frontier shifts from low-granularity annual mappings to higher spatial and temporal resolution designs, as exemplified by the development of a trivariate spectral index for Sentinel-2 capable of delineating recent shifting cultivation clearings on a 20-m grid, enhancing the identifiability of fine-scale disturbances in tropical regions \citep{Li2024}.

The multivariate MCA structure reveals thematic and geographic reorganization of the research ecosystem over 2010--2025 (Figure \ref{fig:mca_temporal}), with 186 categorized observations and a temporal gradient that reconfigures algorithms, contexts, and applications. In 2010--2014, shifting cultivation predominated (26/30) alongside land use and land cover mapping as the dominant application (24/30, 80\%), with low algorithmic diversity in coding (27/30 in \textit{Other}).

In contrast, between 2020 and 2025, the participation of SAT-General contexts expanded (63/114, 55.3\%) together with application diversification, with land use and land cover reduced to 35/114 (30.7\%) and increases in soil (16/114, 14\%), monitoring (10/114, 8.8\%), and yield (9/114, 7.9\%) applications. In parallel, the \textit{DeepLearning} category rose from 2/42 (4.8\%) in 2015--2019 to 24/114 (21.1\%) in 2020--2025, with regional distribution more explicitly mapped to Asia (35/114) and a Global block (50/114), sustaining the operational-readiness interpretation along two complementary dimensions, data governance and reported performance.

\subsection[Functional cluster structure]{Functional cluster structure}

Two cohesive functional groups ($k = 2$, silhouette = 0.215), differentiated by dominant methodological profiles, emerged from hierarchical cluster analysis (Figure \ref{fig:cluster_heatmap}). Cluster 1 (n=70, 37.8\%) is characterized by strong association with \textit{Context=SAT-General} (0.986) and \textit{Algorithm=DeepLearning} (0.30), representing studies that apply advanced computational techniques to generalized traditional agricultural systems, with marked presence in Asia (0.414) and at the global scale (0.343). This cluster demonstrates a statistically significant correlation between deep learning and diversified agricultural contexts ($\rho=0.78$, p$<$0.001), suggesting that algorithmic complexity responds to the need to model nonlinear social-ecological interactions in heterogeneous landscapes. Geographic concentration in Asia (41.4\%) and global studies (34.3\%) corroborates the hypothesis that advanced-technique adoption is associated with greater availability of high-resolution remote sensing data.

Cluster 2 (n=115, 62.2\%) concentrates \textit{Context=Swidden} (0.887) and \textit{Application=LULC} (0.713), reflecting the historical dominance of land-cover mapping in shifting cultivation, with diverse methods (\textit{Algorithm=Other}, 0.809) and hybrid evidence (0.383). This grouping shows methodological asymmetry in which 80.9\% of studies employ conventional algorithms, contrasting with only 7.8\% of deep learning applications. Euclidean-distance analysis in factorial space (MCA) shows this cluster in an orthogonal position relative to Cluster 1 (angle of 87\textdegree{}), indicating complementary but non-overlapping functional profiles.

The dendrogram reveals hierarchical organization of features, where branches aggregating multiple studies were annotated with compact labels indicating dominant dimensions (e.g., \textit{Alg/Other}, \textit{Ctx/SATGen}). The structure suggests non-random category co-occurrence structured by technological affinities, \textit{DeepLearning} in SAT-General contexts and thematic areas such as \textit{LULC} in shifting cultivation, supporting the hypothesis of specialized thematic modules in the analyzed field. The optimal cut height (0.62) confirms robustness of the bipartition, with a cophenetic coefficient of 0.89 validating hierarchical fidelity.

\begin{figure}[H]
\centering
\includegraphics[width=0.65\textwidth]{cluster_heatmap_profiles_edit.png}
\caption{Hierarchical heatmap of functional cluster profiles. \\
\textit{Note} (Rows represent categorical features in the Dimension=Value format, columns represent the 2 identified clusters, and color intensity indicates mean frequency, ranging from 0 when absent to 1 when present in 100\% of the cluster's studies. The lateral dendrogram on the left represents Ward's hierarchy over Euclidean distances among features, and branch annotations identify dominant dimensions, e.g., Alg=algorithm and Ctx=context).}
\label{fig:cluster_heatmap}
\end{figure}

\subsection[Governance and performance]{Governance and performance}

Data governance state, quantified by FAIR compliance, revealed a low-information-density regime in which the mean aggregate score remained at 18.67/100, with marked asymmetry across dimensions, where adherence concentrated in \textit{Findable} (14.78/25, 59.14\%), while \textit{Accessible} (0.16/25, 0.65\%) and \textit{Interoperable} (0.05/25, 0.22\%) remained near zero, and \textit{Reusable} remained limited (3.67/25, 14.67\%) (Figure \ref{fig:fair_radar}).

At indicator level, rich metadata was highly present (97.31\%), contrasting with low explicit licensing (6.45\%) and near-absence of critical auditability artifacts, including available code (1.08\%) and repository data (1.08\%), resulting in only 12.8\% of assessed studies reaching levels deemed adequate for compliance.

\begin{figure}[H]
\centering
\begin{minipage}[t]{0.45\textwidth}
\centering
\includegraphics[width=0.95\linewidth]{fair_radar_only.png}
\par\small (a)
\end{minipage}
\hfill
\begin{minipage}[t]{0.45\textwidth}
\centering
\IfFileExists{fair_indicadores.png}{%
\includegraphics[width=0.95\linewidth]{fair_indicadores.png}%
}{\fbox{Figure not found, fair\_indicadores.png}}
\par\small (b)
\end{minipage}
\caption{Compliance with FAIR data governance principles, with FAIR radar in (a) and per-indicator compliance in (b).}
\label{fig:fair_radar}
\end{figure}

The random-effects meta-analytic synthesis reveals pooled performance above 90\% and explicit between-study heterogeneity in the \textit{forest plot} (Figure \ref{fig:meta_algoritmo}a), where the pooled estimate reached 90.66\% (95\% CI 89.81 to 91.45\%), with heterogeneity quantified by $\tau^2=0.170$, $I\textsuperscript{2}=58.00\%$, and $H^2=2.38$, and dispersion statistically consistent with between-study variance in the $Q=309.71$ test (p=$4.36\times 10^{-17}$).

To assess robustness to sample size imputation, the meta-analysis was repeated excluding 48 studies (37.2\%) without explicit sample sizes. The restricted estimate remained stable at 89.8\% (95\% CI 88.5 to 91.1\%, k=81) compared to 90.7\% in the complete sample (k=129), with similar heterogeneity (I\textsuperscript{2}=61\% vs. 58\%), confirming that imputation did not introduce material bias in the pooled estimate, although individual weights may be imprecise.

Algorithm-family decomposition supports the interpretation of elevated performance, with pooled estimates of 91.33\% for \textit{Random Forest} (95\% CI 89.68 to 92.75, $k=37$) and 90.33\% for SVM (95\% CI 88.84 to 91.65, $k=47$), while \textit{deep learning} architectures preserved 90.51\% (95\% CI 87.08 to 93.09, $k=15$). Meta-regression formalized the temporal component of reported performance over the publication period (Figure \ref{fig:meta_algoritmo}b), with a positive coefficient for centered year (estimate 0.021, SE 0.012, z=1.713, p=0.0866, 95\% CI -0.003 to 0.045), compatible with an increasing reporting trend, though without conclusive evidence at the 5\% level.

\begin{figure}[H]
\centering
\begin{minipage}[t]{0.49\textwidth}
\centering
\includegraphics[width=0.95\linewidth]{meta_analise_algoritmos.png}
\par\small (a)
\end{minipage}
\hfill
\begin{minipage}[t]{0.49\textwidth}
\centering
\includegraphics[width=0.95\linewidth]{meta_regressao_ano.png}
\par\small (b)
\end{minipage}
\caption{Meta-analytic synthesis of accuracies by algorithm accompanied by the meta-regression examining the temporal trend in reported performance over the publication period.}
\label{fig:meta_algoritmo}
\end{figure}

\section[Discussion]{Discussion}\label{sec:discussao}

The evolutionary dynamics of scientific output (Figure \ref{fig:temporal_combined}a) indicate a transition from exploratory decision-support engineering to more robust inference architectures for auditing traditional agricultural systems. This inflection appears in the replacement of support vector machines and random forests by deep neural networks (Figure \ref{fig:temporal_combined}b), consistent with efforts to internalize stochastic biophysical nonlinearity in high-entropy computational models \citep{Weiss2020}.

Performance saturation in discrete classifiers indicates that social-ecological complexity benefits from fuzzy logic-based approaches, where land-use membership is continuous rather than binary, enabling representation of transition-zone ambiguity in shifting cultivation and agroforestry homegardens, as discussed by \citep{Foody2020}, \citep{Osco2021}, and \citep{Denis20151499}.

The co-occurrence network topology (Figure \ref{fig:network_communities}) indicates epistemic compartmentalization relevant to decision-support system development. Segregated modules, one anchored in medium spectral resolutions and another in very high-resolution sensors, support the hypothesis that optical signatures of ecosystem services are scale-dependent, limiting digital-twin generalization across observational regimes and territorial contexts \citep{Basso2020,Jones2020DigitalTwin,VanDerHorn2021DigitalTwin}. This fragmentation indicates that robustness depends on calibrating membership functions for each resolution regime, reducing dimensional incompatibility errors.

The structuring of the vector space in the Multiple Correspondence Analysis (Figure \ref{fig:mca_temporal}) reinforces the premise that the identifiability of edaphic and cultural system states is conditional on the observation technique \citep{Reichstein2019}. Dim1 exhibited an eigenvalue of 0.5685 and 9.17\% of explained inertia, with interpretation supported by high contributions and good representation quality (cos$^2$) for \textit{Context} SAT-General (15.05\%, cos$^2$=0.91) and \textit{Context} Swidden (9.84\%, cos$^2$=0.84), together with association to \textit{Application} LULC (7.84\%, cos$^2$=0.72) and \textit{Algorithm} DeepLearning (11.92\%, cos$^2$=0.54). Conversely, Dim2 exhibited an eigenvalue of 0.3545 and 5.72\% of inertia, dominated by \textit{Evidence} Hyperspectral (28.01\%, cos$^2$=0.75) and a minority component of \textit{Algorithm} Boosting (25.24\%, cos$^2$=0.73), in addition to \textit{Application} Monitoring (14.99\%, cos$^2$=0.57), corroborating that the orthogonal axis captures the transition of instrumental regimes and specialized applications.

The orthogonality between carbon monitoring applications (Cluster 1, Figure \ref{fig:cluster_heatmap}) and slash-and-burn practice mapping (Cluster 2, Figure \ref{fig:cluster_heatmap}) suggests distinct causal mechanisms. Carbon dynamics respond to continuous biophysical variables well captured by deep learning \citep{Celette2016WaterStress}, while delineation of pyrogenic disturbances is governed by discrete topologies. For an auditable decision-support system, this requires hybrid architectures that integrate fuzzy inference for continuous state variables and Boolean logic for discrete event detection \citep{Akram2026}.

Regarding data chain-of-custody integrity, low FAIR compliance (Figure \ref{fig:fair_radar}) creates epistemic opacity incompatible with transparency required for territorial auditing and agroecological transition decisions, because absent provenance metadata and training scripts hinder independent verification of model stability and compromise indicator reliability \citep{Tuia2021,Samuel2021,Ivie2018,Zheng2024}. From a governance-oriented systems-engineering perspective, algorithmic auditability is as necessary as statistical precision \citep{Rudin2019}. Without traceability, inference remains non-verifiable and loses legal or regulatory value for ecosystem-service certification and for design and evaluation of innovations in agricultural systems required by sociotechnical transition frameworks \citep{Tuia2021}.

Under aggregate synthesis, residual meta-analytic variance (Figure \ref{fig:meta_algoritmo}) exposes a precision paradox. Estimated overall accuracy (90.66\%) is high, but substantial heterogeneity ($I^2=58\%$) and diagnostics from Egger's test and \textit{trim-and-fill} indicate reporting asymmetry consistent with optimistic performance under weak spatial validation, a pattern described as spatial overfitting \citep{Ploton2020,Meyer2021}.

Egger's test yielded an intercept of $\beta_0=1.87$ (SE = 0.72; $p=0.009$), indicating asymmetric bias in the standard-error distribution. The \textit{trim-and-fill} adjustment imputed four missing studies, shifting the pooled estimate to 89.12\% (95\% CI 88.17 to 90.03), an absolute difference of -1.54 percentage points relative to the original synthesis. This correction magnitude suggests that publication bias is present but does not drastically alter the substantive conclusion of elevated performance, while still warranting cautious reading in territorial generalization scenarios.

From a fuzzy-systems perspective, this instability reflects attempts to impose crisp categories, such as forest regions and rural areas, on a complex ecological continuum \citep{Fisher2010}. The nominally superior performance of deep architectures does not remove epistemic uncertainty \citep{Hullermeier2021}, but relocates it to latent representations that are harder to interpret \citep{Tuia2021}. The next generation of auditing tools should prioritize algorithmic explainability and explicit uncertainty modeling (confidence intervals, fuzzy sets) over blind maximization of point-accuracy metrics \citep{Rudin2019,Ghilardi2025}.

\subsection[Implications for global sustainability governance]{Implications for global sustainability governance}

At the global governance scale, convergence of data opacity, reporting bias, and epistemic compartmentalization limits operationalization of decision-support systems for monitoring and agroecological transitions in traditional agricultural systems \citep{Wezel2009} within the 2030 Agenda framework, particularly SDG targets~2 (Zero Hunger) and 15 (Life on Land) \citep{UN2015}. Recognized by the GIAHS program in 24 countries and integrated into IPBES assessments on indigenous and local knowledge \citep{Koohafkan2011,IPBES2019}, these systems are nature-society interfaces where monitoring deficits directly compromise design and implementation of innovations in agricultural systems.

Quantitatively, the near-collapse of accessibility (0.65\%) and interoperability (0.22\%) within the aggregate FAIR score of 18.67/100 (Fig.~\ref{fig:fair_radar}) defines the operational magnitude of this limitation. Without reproducible and auditable machine learning pipelines, conversion of remote sensing products into actionable indicators for heritage-site management remains unfeasible under terms required by SDG targets~2.4 and 15.4 \citep{UN2015,Mottet2020TAPE}, which presuppose verifiable monitoring systems as a condition for regulatory compliance. This fragility is exacerbated in Global South communities, where traditional systems sustain livelihoods and agrobiodiversity while open-data infrastructure has lower density and coverage \citep{Kshetri2014DigitalDivide}.

Geographic inequality evidenced by network topology (Fig.~\ref{fig:network_communities}) and by the factorial structure of Multiple Correspondence Analysis (Fig.~\ref{fig:mca_temporal}) adds risk of algorithmic coding bias with direct implications for external validity. As documented in Cluster~1 (Asia 41.4\%; global scale 34.3\%), systematic underrepresentation of traditional agricultural systems in Africa and Central Asia means that spectral and phenological patterns encoded in data-rich settings are not generalizable, amplifying classification errors in underrepresented landscapes \citep{Santos2007Epistemologies,Tuia2021}. Reducing this asymmetry requires interoperable data infrastructures that reduce digital exclusion barriers \citep{Kshetri2014DigitalDivide} and participatory validation protocols with local knowledge holders, a necessary condition for external robustness in regulatory and heritage-management contexts \citep{Berkes2003,Altieri2004}.

At the science-policy interface, transition from static classifiers to adaptive and explainable decision-support systems requires governance mechanisms that ensure algorithmic traceability, community consent, and demonstrated inter-regional transferability. Without these conditions, predictive capacity does not translate into an auditable management instrument. This technical and institutional integration is necessary for machine learning to operate as a monitoring and assessment tool with regulatory value for safeguarding social-ecological integrity of traditional agricultural systems under accelerating global change.

\subsection[Limitations]{Limitations}

Regarding limitations, the Egger intercept ($\beta_0 = 1.87$; $p = 0.009$) and the \textit{trim-and-fill} correction ($-$1.54 pp.) classify the pooled accuracy of 90.66\% as an upper bound on \textit{reported} performance, not as an operational benchmark; the sensitivity analysis excluding the 37.2\% of studies with imputed sample size confirmed estimate stability (89.8\% vs. 90.7\%), indicating that imputation did not introduce material bias. The absence of formal study-level risk-of-bias instruments (PROBAST, QUADAS-2) prevents stratification by methodological quality, while the restriction to English-language terms in Scopus and Web of Science may underrepresent monitoring initiatives in non-Anglophone regions. Finally, spatial validation strategies were not systematically extracted, limiting the empirical quantification of performance degradation in external validity. Taken together, the findings provide a diagnostic reference framework of the current state rather than deployment-ready regulatory benchmarks.

\section[Conclusions]{Conclusions}

This study systematizes Traditional Agricultural Systems as coupled social-ecological systems whose typicality emerges from nonlinear interactions between edaphoclimatic variables and cultural practices. Results are consistent with emergence of a rapidly expanding digital monitoring paradigm during 2010--2025, anchored in integration of remote sensors and deep learning architectures. The knowledge network topology, with density 0.3455, reveals a cohesive yet geographically unequal research ecosystem, where concentration of studies in specific regions coexists with critical validation gaps in traditional system contexts in Africa, Central Asia, and the Pacific.

Examination of the 18 high-adherence studies suggests that part of the frontier has advanced beyond simple land-cover classification toward digital twins with potential to infer phytopathologies, carbon dynamics, and human mobility under reported operational conditions. However, translating this evidence into safeguarding decisions aligned with global sustainability frameworks, including management of GIAHS heritage sites and SDG targets~2.4 and 15.4, still depends on external robustness, because models lack systematic inter-regional validation across diverse biophysical and cultural contexts where these systems operate.

In sum, operational applicability of machine learning in traditional agricultural system contexts is not limited to algorithmic accuracy, but depends on FAIR-compliant data infrastructures, rigorous spatial and temporal validation protocols, and participatory governance mechanisms that ensure algorithmic transparency and epistemic inclusiveness across regions. Future research should consolidate the transition from static classifiers to adaptive, auditable, and globally transferable systems, with potential to convert terabytes of orbital data into verifiable evidence for sustainable management of agricultural heritage landscapes at global scale.

\section*{Acknowledgments}

The authors thank the Universidade Federal de Sergipe (UFS), the Universidade Estadual de Feira de Santana (UEFS), and the Instituto Federal de Sergipe (IFS) for the institutional and infrastructural support that enabled this research.

\section*{Declarations}

\subsection*{Funding}

The authors did not receive funding from any organization for the submitted work. No grants, financial support, or other resources were received.

\subsection*{Conflicts of interest}

The authors declare no relevant financial or non-financial interests to disclose.

\subsection*{Ethics approval}

Not applicable. This study is based exclusively on published secondary data and did not involve human participants, personal data, or biological material.

\subsection*{Consent to participate}

Not applicable.

\subsection*{Consent for publication}

Not applicable.

\subsection*{Data availability}

The complete dataset supporting this study, including the bibliographic corpus, analysis scripts, and intermediate results, is publicly available at the Open Science Framework (OSF) under DOI \url{https://doi.org/10.17605/OSF.IO/J7STC}.

\subsection*{Code availability}

All analysis scripts (Python and R) used for automated screening, meta-analysis, network analysis, multiple correspondence analysis, and FAIR compliance assessment are publicly available at the Open Science Framework (OSF) under DOI \url{https://doi.org/10.17605/OSF.IO/J7STC}.

\subsection*{Authors' contributions}

Conceptualization, L.D.V.S. and C.V.S.O.; Methodology, L.D.V.S.; Software, L.D.V.S.; Formal analysis, L.D.V.S.; Investigation, C.V.S.O. and L.D.V.S.; Data Curation, L.D.V.S.; Writing (original draft), L.D.V.S. and C.V.S.O.; Writing (review \& editing), P.R.G., R.N.A.F., T.C.A., G.S.Q. and F.S.R.H.; Visualization, L.D.V.S.; Supervision, F.S.R.H. and P.R.G.; Project administration, C.V.S.O. All authors read and approved the final manuscript.

\bibliography{referencias}

\section*{References of the Meta-Analysis}

The complete list of 244 studies included in this meta-analysis is provided as a separate supplementary document (\textit{References of the Meta-Analysis}), available alongside this manuscript and deposited at the Open Science Framework (OSF) under DOI \url{https://doi.org/10.17605/OSF.IO/J7STC}.

\end{document}

